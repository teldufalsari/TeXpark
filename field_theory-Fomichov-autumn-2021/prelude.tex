%%% Работа с русским языком
\usepackage{cmap}					          % поиск в PDF
\usepackage[T2A]{fontenc}			      % кодировка
\usepackage[utf8]{inputenc}               % кодировка исходного текста
\usepackage[english, russian]{babel}   % локализация и переносы
\usepackage{breqn}


%%% Страница 
\usepackage{extsizes} % Возможность сделать 14-й шрифт
\usepackage{geometry}  
\geometry{left=20mm,right=20mm,top=25mm,bottom=30mm} % задание полей текста

\usepackage{titleps}      % колонтитулы
\newpagestyle{main}{
	\setheadrule{.4pt}                      
	\sethead{\CourseName}{}{\hyperlink{intro}{\;Назад к содержанию}}
	\setfootrule{.4pt}                       
	\setfoot{\CourseDate \; ФРКТ МФТИ}{}{\thepage} 
}      
\pagestyle{main}    % Устанавливает контитулы на странице


%%%  Текст (Тут вообще всё можно настроить как надо, но вообще стандарт есть стандарт)
\setlength\parindent{0pt}         % Устанавливает длину красной строки 0pt (А я не уверен, что это нужно (!))
\sloppy                                        % строго соблюдать границы текста
\linespread{1.3}                           % коэффициент межстрочного интервала
\setlength{\parskip}{0.5em}                % вертик. интервал между абзацами
%\setcounter{secnumdepth}{0}                % отключение нумерации разделов (Я бы не стал трогать, но вообще почему нет)
\usepackage{multicol}				          % Для текста в нескольких колонках (Можно сразу и multirow подключить)
%\usepackage{soul}
\usepackage{soulutf8} % Модификаторы начертания


%%% Гиперссылки
\usepackage{hyperref}
\usepackage[usenames,dvipsnames,svgnames,table,rgb]{xcolor}
\hypersetup{				% Гиперссылки
	unicode=true,           % русские буквы в раздела PDF\\
	pdfstartview=FitH,
	pdftitle={Заголовок},   % Заголовок
	pdfauthor={Автор},      % Автор
	pdfsubject={Тема},      % Тема
	pdfcreator={Создатель}, % Создатель
	pdfproducer={Производитель}, % Производитель
	pdfkeywords={keyword1} {key2} {key3}, % Ключевые слова
	colorlinks=true,       	% false: ссылки в рамках; true: цветные ссылки
	linkcolor=blue,          % внутренние ссылки
	citecolor=green,        % на библиографию
	filecolor=magenta,      % на файлы
	urlcolor=NavyBlue,           % на URL
}

%%% Делать римские цифры:
\newcommand{\Romannum}[1]{\uppercase\expandafter{\romannumeral #1\relax}}

%%% Для формул
\usepackage{amsmath}          
\usepackage{amssymb}
\usepackage{cases}
\usepackage{gensymb}
\usepackage{extarrows}

%%% Дополнительные математические операции
\DeclareMathOperator{\re}{Re}
\DeclareMathOperator{\im}{Im}
\DeclareMathOperator{\Arg}{Arg}
\DeclareMathOperator{\Ln}{Ln}
\DeclareMathOperator{\Div}{div}
\DeclareMathOperator{\grad}{grad}
\DeclareMathOperator{\rot}{rot}
\DeclareMathOperator{\arsh}{arsh}
\DeclareMathOperator{\arch}{arch}
\DeclareMathOperator{\cum}{cum}
\DeclareMathOperator{\Cum}{Cum}
\DeclareMathOperator{\trace}{trace}


%%%%%% theorems
\usepackage{amsthm}  % for theoremstyle

\theoremstyle{plain} % Это стиль по умолчанию, его можно не переопределять.
\newtheorem{theorem}{Теорема}[section]
\newtheorem{prop}[theorem]{Утверждение}
\newtheorem{lemma}{Лемма}[section]
\newtheorem{sug}{Предположение}[section]

\theoremstyle{definition} % "Определение"
\newtheorem{Def}{Определение}
\newtheorem{corollary}{Следствие}[theorem]
\newtheorem{problem}{Задача}[section]

\theoremstyle{remark} % "Примечание"
\newtheorem*{nonum}{Решение}
\newtheorem*{defenition}{Def}
\newtheorem*{example}{Пример}
\newtheorem*{note}{Замечание}


%%% Работа с картинками
\usepackage{graphicx}                           % Для вставки рисунков
\graphicspath{{images/}{images2/}}        % папки с картинками
\setlength\fboxsep{3pt}                    % Отступ рамки \fbox{} от рисунка
\setlength\fboxrule{1pt}                    % Толщина линий рамки \fbox{}
\usepackage{wrapfig}                     % Обтекание рисунков текстом
\graphicspath{{images/}}                     % Путь к папке с картинками

\newcommand{\drawsome}[1]{            % Для быстрой вставки картинок
	\begin{figure}[h!]
		\centering
		\includegraphics[scale=0.7]{#1}
		\label{fig:first}
	\end{figure}
}
\newcommand{\drawsomemedium}[1]{
	\begin{figure}[h!]
		\centering
		\includegraphics[scale=0.45]{#1}
		\label{fig:first}
	\end{figure}
}
\newcommand{\drawsomesmall}[1]{
	\begin{figure}[h!]
		\centering
		\includegraphics[scale=0.3]{#1}
		\label{fig:first}
	\end{figure}
}

%%% Графека
\usepackage{tikz}
\usetikzlibrary{angles, babel}


%%% облегчение математических обозначений
\newcommand{\R}{\mathbb{R}}
\newcommand{\N}{\mathbb{N}}
\newcommand{\Cc}{\mathbb{C}}
\newcommand{\Cex}{\overline{\mathbb{C}}}
\newcommand{\Z}{\mathbb{Z}}
\newcommand{\E}{\mathbb{E}}
\newcommand{\Dh}{\mathfrak{D}}
\newcommand{\phee   }{\varphi}
\newcommand{\pard}[2]{\frac{\partial#1}{\partial#2}}
\newcommand{\pardd}[2]{\frac{\partial^2#1}{\partial{#2}^2}}
\newcommand{\brackets}[1]{\left({#1}\right)}      % автоматический размер скобочек
\newcommand{\braces}[1]{\left\{{#1}\right\}}
\newcommand{\sqbrk}[1]{\left[{#1}\right]}
\newcommand{\abs}[1]{\left|{#1}\right|}
\newcommand{\cnj}{\overline}
\newcommand{\Relroot}{\sqrt{1 - V^2 \big/ c^2}}
\newcommand{\relroot}{\sqrt{1 - v^2 \big/ c^2}}
\newcommand{\vecp}[2]{\sqbrk{\vec{#1}\times\vec{#2}}}
\newcommand{\wecp}[2]{\sqbrk{\vec#1\times\vec#2}}
\newcommand{\dotp}[2]{\brackets{\vec{#1}\cdot\vec{#2}}}
\newcommand{\sotp}[2]{\brackets{\vec#1\cdot\vec#2}}
\newcommand{\lch}{e_{\alpha\beta\gamma}}
\newcommand{\vol}{\mathcal{V}}
\newcommand{\erg}{\mathcal{E}}
\newcommand{\mixd}[3]{\frac{\partial^2#1}{\partial#2\partial#3}}
\newcommand{\med}[1]{\overline{#1}}
\newcommand{\medv}[1]{\overline{\vec{#1}}}
\newcommand{\ynt}{\iiiint\limits_{-\infty}^{+\infty}}
\newcommand{\eent}{\int\limits_{-\infty}^{+\infty}}
\newcommand{\vt}{\bigm|_{t'}}
% Здесь можно добавить ваши индивидуальные сокращения

