\newpage
\section{Излучение релятивистских частиц}
\subsection{Потенциалы Лиенара-Вихерта}
    Рассмотрим движение одной частицы с произвольной скоростью без ограничения $v \ll c$. Воспользуемся выражением для векторного потенциала в
    следующей форме:
    \begin{equation}
        \vec{A}(\vec{r}, t) = \frac 1c \iiiint G^R(\vec{r} - \vec{r'}, \: t - t') \vec{j}(\vec{r'}, t')d\vol'dt', \label{vec_pot_rel}
    \end{equation}
    
    где плотность тока и функция Грина заданы известными выражениями:
    \begin{gather*}
        \vec{j}(\vec{r'}, t') = e\vec{v}(t')\delta(\vec{r'} - \vec{r}(t')),\\
        G^R(\vec{r}, \vec{r'}, t, t') = \frac{\delta\brackets{t - t' - \frac{\abs{\vec{r} - \vec{r'}}}{c}}}{\abs{\vec{r} - \vec{r'}}}.
    \end{gather*}
    Проинтегрируем уравнение (\ref{vec_pot_rel}) по объёму:
    \begin{gather*}
        \vec{A}(\vec{r}, t) = \frac 1c \iiiint \frac{\delta\brackets{t - t' - \frac{\abs{\vec{r} - \vec{r'}}}{c}}}{\abs{\vec{r} - \vec{r'}}}
        e\vec{v}(t')\delta(\vec{r'} - \vec{r}(t')) d\vol'dt' = \\ =
        \frac 1c \int \frac{\delta\brackets{t - t' - \frac{\abs{\vec{r} - \vec{r}(t')}}{c}}}{\abs{\vec{r} - \vec{r}(t')}} r\vec{v}(t')dt'.
    \end{gather*}
    Величина $\vec{r}$, как и прежде, есть радуис-вектор точки наблюдения, а $\vec{r}(t')$ --- положение частицы в момент $t'$ (а $\vec{r}(t)$ --- траектория частицы).
    Введём вспомогательные функции $\vec{R}(t') = \vec{r} - \vec{r}(t')$, $g(t') = t' + \frac{R(t')}{c}$. Преобразуем полученное уравнение:
    \begin{gather*}
        \vec{A}(\vec{r}, t) = \int \frac{e \vec{v}(t')}{cR(t')} \delta\brackets{t - t' \frac{R(t')}{c}} dt'= \\ =
        \int \frac{e\vec{v}(t')}{cR(t')}\delta(t - g(t')) dt' = \int \frac{e\vec{v}(t')}{cR(t')}\delta(t - g)\frac{dg}{\dot{g}(t')}; \\
        \dot{g}(t') = 1 - \frac{\sotp{R(t')}{v(t')}}{cR(t')} \equiv \left. \brackets{1 - \frac{\dotp{n}{v}}{c}} \right|_{t'},
    \end{gather*}
    где $\vec{n} = \vec{n}(t') = \frac{\vec{R}(t')}{R(t')}$. Окончательно для векторного потенциала получаем
    \begin{equation}
        \vec{A}(\vec{r}, t) = \left. \frac{e\vec{v}}{cR(t') \brackets{1 - \frac{\dotp{n}{v}}{c}}} \right|_{t'}, \label{LV_A}
    \end{equation}
    где $t'$ находится из уравнения
    \begin{equation}
        t = g(t') = t' + \frac{R(t')}{c} \label{rel_abs_time}
    \end{equation}
    в котором $t$ --- время, связанное с наблюдателем. 
    \begin{figure}[t]
        \centering{
            \begin{tikzpicture}
    \filldraw [black] (0,0) circle (2pt);
    \draw [->] (0,0) -- (-1, 3);
    \node [] at (-0.4, 1.6) {$\vec{r}$};
    \filldraw [blue] (-1.02, 3.06) circle (2pt);
    \node [anchor = west] at (0,0) {$0$};
    \draw [->] (0, 0) -- (-1, -2);
    \node [] at (-0.1, -1.1) {$\vec{r}(t')$};
    \draw [brown] (-1, -2) arc (50:100:3);
    \draw [brown] (-1, -2) arc (230:250:3);
    \draw [->] (0, 0) -- (-2.5, -1.34);
    \node[] at (-1.4, -0.35) {$\vec{r}(t)$};
    \draw [dashed] (-2.5, -1.34) -- (-1, 3);
    \draw [dashed] (-1, -2) -- (-1, 3);
    \draw [->] (-1, -2) -- (-1, -1.4) node [anchor = east] {$\vec{n}(t')$};
    \draw [->] (-2.5, -1.34) -- (-2.361, -0.95) node [anchor = south east] {$\vec{n}(t)$};
    \node [] at (1.6, -2.5) {$\vec{r}(t)$ --- траектория};
\end{tikzpicture}
        }
        \caption{К выводу потенциалов Лиенара-Вихерта}
    \end{figure}
    Аналогичным образом получается формула скалярного потенциала релятивистской частицы:
    \begin{equation}
        \phee(\vec{r}, t) = \left. \frac{e}{R(t')\brackets{1 - \frac{\dotp{n}{v}}{c}}} \right|_{t'}. \label{LV_F}
    \end{equation}
    Формулы (\ref{LV_A}) и (\ref{LV_F}) называют \textit{потенциалами Лиенара-Вихерта}. Вместе с системой уравнений
    \begin{gather*}
        \vec{R}(t') = \vec{r} - \vec{r}(t')\\
        t = t' + \frac{R(t')}{c}\\
        \vec{n}(t) = \frac{\vec{R}(t')}{R(t')}
    \end{gather*}
    они описывают излучение отдельной релятивисткой частицы. Зная потенциалы, можно найти величины полей: $\vec{H} = \rot\vec{A}$, $\vec{E} = -\frac 1c \pard{\vec{A}}{t} - \nabla\phee$.
    Вывод этих формул отличается крайней громоздкостью, поэтому не приводится и (скорее всего) не спрашивается на экзамене. Да и сами формулы не самые компактные:
    \begin{gather*}
        \vec{E}(\vec{r}, t) = \left. \frac{e\sqbrk{\vec{n} \times \sqbrk{\brackets{\vec{n} - \frac{v}{c}} \times \vec{w}}}}{c^2R\brackets{1 - \frac{\dotp{n}{v}}{c}}^3} \right|_{t'} +
        \left. \frac{e \brackets{1 - \frac{v^2}{c^2}} \brackets{\vec{n} - \frac{\vec{v}}{c}}}{R^2 \brackets{1 - \frac{\dotp{n}{v}}{c}}^3} \right|_{t'} \\
        \vec{H}(\vec{r}, t) = \wecp{n(t')}{E(\vec{r}, t)}
    \end{gather*}
    Здесь $\vec{w} = \dot{\vec{v}}$ --- ускорение частицы. Обратим внимание, что второе слагаемое при $v = 0$ превращается в закон Кулона.

\subsection{Поле на больших расстояниях}
    Рассмотрим поле на расстоянии от точки, достаточно большом, чтобы <<кулоновское>> слагаемое можно было считать пренебрежимо малым:
    \begin{equation*}
        \vec{E}(\vec{r}, t) = \left. \frac{e\sqbrk{\vec{n} \times \sqbrk{\brackets{\vec{n} - \frac{v}{c}} \times \vec{w}}}}{c^2R\brackets{1 - \frac{\dotp{n}{v}}{c}}^3} \right|_{t'}
    \end{equation*}
    Найдём угловое распределение такого излучения:
    \begin{gather*}
        \frac{dI}{d\Omega} = c\frac{E^2}{4\pi}R^2 = \frac{e^2}{4\pi c^3}
        \frac{\sqbrk{\vec{n} \times \sqbrk{\brackets{\vec{n} - \frac{v}{c}} \times \vec{w}}}^2}{\brackets{1 - \frac{\dotp{n}{v}}{c}}^6}.
    \end{gather*}
    При $v \ll c$ получаем уже знакомую формулу:
    \begin{equation*}
        \frac{dI}{d\Omega} \approx \frac{e^2}{4\pi c} \wecp{n}{w}^2 = \frac{e^2w^2\sin^2\theta}{4\pi c^3} = \frac{\ddot{\vec{d}}^2\sin^2\theta}{4\pi c^3}.
    \end{equation*}
    Пусть теперь $v \approx c$, тогда угол между векторамb $\vec{n}$ и $\vec{v}$ мал: $\vartheta \ll 1$. Следовательно,
    \begin{gather*}
        1 - \frac{\dotp{n}{v}}{c} = 1 - \frac{v \cos\vartheta}{c}  \approx 1 - \frac vc \brackets{1 - \frac{\vartheta^2}{2}} = 1 - \frac vc + \frac vc \frac{\vartheta^2}2,
    \end{gather*}
    Далее:
    \begin{gather*}
        \gamma^2 = \frac{1}{1 - v^2 / c^2} \approx \frac{1}{2\brackets{1 - \frac vc}} \: \Rightarrow \: 1 - \frac vc \approx \frac{1}{2\gamma},
    \end{gather*}
    и, с учётом этого,
    \begin{equation*}
        1 - \frac{\dotp{n}{v}}{c} \approx \frac 12 \brackets{\gamma^{-2} + \vartheta^2}.
    \end{equation*}
    Пусть также $\vec{v} \parallel \vec{w}$ или $\angle(\vec{v}, \vec{w}) \approx 0$. Окончательно для углового распределения получаем:
    \begin{equation*}
        \frac{dI}{d\Omega} = \frac{e^2}{4\pi c^3} \frac{w\vartheta^22^6}{\brackets{\gamma^{-2} + \vartheta^2}^6} = \frac{16e^2}{\pi c^3} \frac{w\vartheta^2}{\brackets{\gamma^{-2} + \vartheta^2}^6}
    \end{equation*}
    Функция $\frac{dI}{d\Omega}(\vartheta)$ ведёт себя очень резко в области малых углов $\vartheta$. Она имеет провал при $\vartheta = 0$ и максимумы при
    $\vartheta = \pm \frac{1}{\sqrt{5}\gamma}$. Основное излучение испускается в узкий конус шириной порядка $\frac{1}{\gamma}$.

\subsection{Мощность излучения}
    Порция энергии, излучаемая в телесный угол $d\Omega$ за время $dt$, воспринимаемая в точке наблюдения, может быть записана как
    \begin{equation*}
        -d^2\erg = dI \cdot dt.
    \end{equation*}
    Однако скорость потерь энергии частицы можно связать с собственным временем частицы: энергия, полученная наблюдателем в промежуток времени от $t$ до $t + dt$ была
    испущена частицей за время от $t'$ до $t' + dt'$. Таким образом,
    \begin{equation*}
        -\frac{d^2\erg}{dt'd\Omega} = \frac{dI}{d\Omega} \cdot \frac{dt}{dt'}.
    \end{equation*}
    Выразим отсюда скорость потерь энергии:
    \begin{equation*}
        -\frac{d\erg}{dt'} = \int d\Omega\cdot\frac{dI}{d\Omega}\cdot\frac{dt}{dt'}.
    \end{equation*}
    В силу уравнения (\ref{rel_abs_time}) получаем
    \begin{equation*}
        \frac{dt}{dt'} = 1 - \frac{\dotp{n}{v}}{c},
    \end{equation*}
    откуда после подстановки формулы углового распределения излучения имеем
    \begin{equation*}
        -\frac{d\erg}{dt'} = \int d\Omega \frac{e^2}{4\pi c^3} \frac{\sqbrk{\vec{n} \times \sqbrk{\brackets{\vec{n} - \frac{\vec{v}}{c}} \times \vec{w}}}^2}
        {\brackets{1 - \frac{\dotp{n}{v}}{c}}^5}.
    \end{equation*}
    Заметим, что величина $-\frac{d\erg}{dt'}$ является релятивистским инвариантом. В системе $K = K_0$ (системе покоя частицы) для неё справедливо нерелятивистское
    приближения, которое для покоящейся частицы совпадает с точным решением:
    \begin{equation}
        - \frac{d\erg}{dt'} = \int d\Omega \frac{e^2}{4\pi c^3} w_0^2\sin^2\theta = \frac{2e^2w_0^2}{3c^3} = - \frac{d\erg_0}{dt_0'}, \label{rel_rad_self}
    \end{equation}
    индекс <<$0$>> обозначает величины в системе $K_0$. Рассмотрим 4-вектор потерь энергии-импульса частицы:
    \begin{gather*}
        dp^i = \brackets{-\frac{d\erg}{c},\: -d\vec{p}}, \\
        \left. dp^i \right|_{K = K_0} = \brackets{-\frac{d\erg_0}{c},\: d\vec{p}_0}.
    \end{gather*} 
    Заметим, что в $K_0$-системе $dI \sim \sin^2\theta d\theta$, а это выражение симметрично относительно замены $\theta \:\rightarrow\: \pi - \theta$.
    То есть, в одном направлении излучается то же количество энергии, что и в противоположном, и, как следствие, $dp_0 = 0$, и
    $\left. dp^i \right|_{t'} = \brackets{-\frac{d\erg_0}{c}, \:0}$ Переходя в произвольную $K$-систему,
    \begin{equation*}
        d\erg = \frac{d\erg_0}{\relroot} \:\Rightarrow\: -\frac{d\erg}{dt'} = -\frac{d\erg_0}{dt'\relroot},
    \end{equation*}
    и в силу того, что собственное время частицы $d\tau \equiv dt'_0 = dt'\relroot$, получаем утверждение об инвариантности потерь энергии в системе покоя:
    \[
        -\frac{d\erg}{dt'} = -\frac{d\erg_0}{dt_0'}.
    \]
    Учитывая уравнение (\ref{rel_rad_self}), приходим к \textit{формуле Лармора}
    \begin{equation*}
        \boxed{-\frac{d\erg}{dt'} = \frac{2e^2w_0^2}{3c^3}}
    \end{equation*}
    Энергию потерь можно выразить через величины текущей системы отсчёта, использую свойство (\ref{four_acc_feature}) 4-ускорения:
    \begin{equation*}
        -\frac{d\erg}{dt'} = -\frac{2e^2c}{3}w^iw_i.
    \end{equation*}
    В явном виде квадрат 4-ускорения равен
    \begin{gather*}
        w^iw_i = \frac{\gamma^8}{c^6}\dotp{v}{w}^2 - \frac{\gamma^8}{c^8}v^2\dotp{v}{w}^2 - 2\frac{\gamma^6}{c^6} \dotp{v}{w}^2 - \frac{\gamma^4}{c^4}w^2 = \\ =
        \frac{\gamma^8}{c^6}\dotp{v}{w}\brackets{1 - \frac{v^2}{c^2}} - 2\frac{\gamma^6}{c^6}\dotp{v}{w}  - \frac{\gamma^4}{c^4}w^2 = \\ =
        -\frac{\gamma^6}{c^4}\brackets{w^2\gamma^2 + \frac{1}{c^2}\dotp{v}{w}^2} = \\ =
        -\frac{\gamma^6}{c^4}\brackets{w^2\brackets{1 - \frac{v^2}{c^2}} + \frac{1}{c^2}\dotp{v}{w}^2} = \\ =
        -\frac{\gamma^4}{c^4}\brackets{w^2 - \frac{1}{c^2}\brackets{v^2w^2 - \dotp{v}{w}^2}} = \\ =
        -\frac{\gamma^4}{c^4}\brackets{w^2 - \frac{\vecp{v}{w}^2}{c^2}}.
    \end{gather*}
    Мощность излучения в системе покоя равна
    \begin{equation*}
        -\frac{d\erg}{dt'} = \frac{2e^2\gamma^6}{3c^3}\brackets{w^2 - \frac{\vecp{v}{w}^2}{c^2}}.
    \end{equation*}
    Ожидаемо при $v \ll c$ получаем нерелятивистскую формулу
    \[
        -\frac{d\erg}{dt'} = \frac{2e^2w^2}{3c^3}.
    \]

\newpage
\subsection{Синхротронное излучение}
    \begin{figure}[h]
        \centering{
            \begin{tikzpicture}
    \draw [dashed] (0,0) circle (2);
    \filldraw [blue] (0,0) circle (1pt);
    \draw [blue] (0,0) circle (3pt);
    \node [blue, anchor = south west] at (0,0) {$\vec{H}$};
    \draw [thin, ->] (0,0) -- (0.7247, -1.864);
    \node [] at (0.53, -0.8) {$R$};
    \filldraw [red] (0, 2) circle (2pt);
    \draw [->] (0, 2) -- (-1, 2) node [anchor = south east] {$\vec{v}$};
\end{tikzpicture}
        }
    \end{figure}
    Применим полученные формулы к частице, движущейся по окружности в магнитном поле, считая, что потери на излучение малы и не влияют на движение частицы.
    \begin{gather}
        -\frac{d\erg}{dt'} = \frac{2e^2\gamma^6}{3c^3}\brackets{w^2 - \frac{v^2w^2}{c^2}} = \notag \\ = 
        \frac{2e^2\gamma^2w^2}{3c^3}\brackets{1 - \frac{v^2}{c^2}} = \frac{2e^2\gamma^4w^2}{3c^3}. \label{smol_losses}
    \end{gather}
    Уравнение движения частицы при малых потерях
    \begin{equation}
        \frac{d\vec{p}}{dt} = \frac ec \vecp{v}{H}, \: \abs{p} = const \label{lossless_motion}
    \end{equation}
    С учётом релятивисткой формулы для импульса
    \[
        \vec{p} = \frac{m\vec{v}}{\relroot} = m\gamma\vec{v}
    \]
    уравнение (\ref{lossless_motion}) примет вид
    \begin{equation*}
        m\gamma\vec{w} = \frac ec \vecp{v}{H} \: \Rightarrow \: w^2 = \frac{e^2v^2H^2}{m^2c^2\gamma^2}.
    \end{equation*}
    Подставим квадрат ускорения в формулу (\ref{smol_losses}):
    \begin{equation*}
        -\frac{d\erg}{dt'} = \frac{2e^4\gamma^2v^2H^2}{3m^2c^5}.
    \end{equation*}
    Так как радиус обращения частицы равен
    \begin{equation*}
        R = \frac{cp_{\perp}}{|e|H} = \frac{mc\gamma v}{|e|H} \overset{v \approx c}{\approx} \frac{mc^2\gamma}{|e|H},
    \end{equation*}
    то мощность излучения можно записать в виде
    \begin{equation*}
        -\frac{d\erg}{dt'} \approx \frac{2\gamma^4}{3}c\frac{e^2}{R^2}.
    \end{equation*}

    Обсудим критерий применимости приближения малых потерь: если период обращения частицы равен $T$, то условие малости потерь
    можно сформулировать как
    \begin{equation}
        \abs{\frac{d\erg}{dt'}}\cdot T \ll \erg. \label{smollness_criterion}
    \end{equation}
    Далее, учитывая, что
    \begin{equation*}
        T = \frac{2\pi}{\omega} = \frac{2\pi}{eHc} = \frac{2\pi\erg}{eHc},
    \end{equation*}
    преобразуем условие (\ref{smollness_criterion}):
    \begin{gather*}
        \frac{2e^4\gamma^2v^2H^2}{3m^2c^5} \cdot \frac{2\pi\erg}{eHc} \ll \erg \: \Rightarrow \: \gamma\frac{v}{c} \ll \sqrt{\frac{H}{H_0}},
    \end{gather*}
    где $\displaystyle H_0 \sim \frac{m^2c^4}{e^3} \sim 0,6 \cdot 10^{12} \:\textrm{Тл}$ . Для полей порядка 1 Тл допустимая величина $\gamma \ll 10^6$,
    что соответствует энергии частицы $\erg \ll 10^{12} \: \textrm{эВ} = 1 \:\textrm{ТэВ}$.

    \begin{figure}[t]
        \centering{
            \begin{tikzpicture}
    \draw [dashed, gray] (0,0) circle (2);
    \filldraw [blue] (0,0) circle (1pt);
    \coordinate (c1) at (1.4142, 1.4142);
    \coordinate (a1) at (3.227, 0.569);
    \coordinate (b1) at (2.259, -0.398);
    \coordinate (d1) at (2.8284, 0);
    \coordinate (c2) at (1.732, -1);
    \coordinate (a2) at (1.385, -2.97);
    \coordinate (b2) at (0.2, -2.286);
    \coordinate (d2) at (0.732, -2.732);
    \draw [thin, gray, -] (0, 0) -- (c1);
    \draw [thin, gray, -] (0, 0) -- (c2);
    \draw[thin, red, ->] (c1) -- (d1);
    \draw[thin, red, dashed] (c1) -- (b1);
    \draw[thin, red, dashed] (c1) -- (a1);
    \pic [draw, angle radius = 16mm] {angle =  b1--c1--d1};
    \pic [draw, angle radius = 18mm] {angle =  d1--c1--a1}; \node[anchor = north] at (a1) {$\theta$};
    
    \coordinate (o) at (0,0);
    \draw[thin, red, dashed] (c2) -- (b2);
    \draw[thin, red, dashed] (c2) -- (a2);
    \draw [thin, red, ->] (c2) -- (d2);
    \pic [draw, angle radius = 16mm] {angle =  b2--c2--d2};
    \pic [draw, angle radius = 18mm] {angle =  d2--c2--a2};
    \node [anchor = north west] at (b2) {$\theta$};
    \pic [draw, angle radius = 5mm] {angle = c2--o--c1};
    \node [] at (0.65, 0)  {$\vartheta$};
    \node [anchor = south west] at (c1) {$t$};
    \node [anchor = north west] at (c2) {$t + \vartheta / \omega$};
\end{tikzpicture}
        }
        \caption{К определению длины когерентности. Конусы перекрываются, когда $\vartheta \lesssim \theta$.}
    \end{figure}
    В пункте №2 было установлено, что при $v \sim c$ частица излучает в узкий конус угловой шириной $\vartheta \sim 1/\gamma$.
    Будем наблюдать излучение частицы в некоторой удалённой точке. В направлении на точку наблюдения частица излучает не мгновенно,
    а некоторый промежуток времени, в течении которого её конус излучения <<захватывает>> это направление. Ясно, что за это время частица
    проходит угол $\theta$ по окружности. Длина отрезка, который частица проходит за это время
    \[
        l = R\phee = R\theta \sim\frac{E}{\gamma}
    \]
    называют \textit{длиной когерентности}. Посмотрим, как изменяется фаза волны за это время:
    \begin{gather*}
        \phee = \omega t - \dotp{k}{r}, \:\: \vec{k} = \frac{\omega}{c}\vec{r};\\
        \Delta\varphi = \omega\Delta t - \brackets{\vec{k} \cdot \Delta\vec{r}} = \omega\Delta t\brackets{1 - \frac{v}{c}} = \\ =
        \omega\frac{l}{c}\cdot\frac{1}{2\gamma^2} = \frac{\omega R}{2\gamma^3 c}.
    \end{gather*}
    Здесь учтено, что $\abs{\Delta\vec{r}} \ll R$ и , следовательно, $\vec{k} \sim \vec{n} \parallel \Delta \vec{r}$. Если $\Delta\phee > 2\pi$, то
    волны с разных участков уничтожают друг друга --- происходит деструктивная интерференция. Для того, чтобы излучение было наблюдаемым, необходимо
    \[
        \frac{\omega R}{2\gamma^3c} \lesssim 2\pi \: \Rightarrow \: \omega \sim \frac{c}{R}\gamma^3.
    \]
    Подставим радиус обращения частицы $R = \frac{mc^2\gamma}{\abs{e}H}$:
    \begin{equation*}
        \omega \sim \frac{\abs{e}H}{mc}\gamma^2.
    \end{equation*}
    Эта формула выражает зависимость спектрального максимума синхротронного излучения от заданного магнитного поля.
    