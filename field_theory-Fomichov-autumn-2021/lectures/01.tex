\newpage
\section{Принцип относительности. Преобразования Лоренца}
\subsection{Преобразования Лоренца}

Основные положения специальной теории относителььности:
\begin{enumerate}
    \item Инерциальные системы отсчёта существуют;
    \item Пространство-время обладает однородностью и изотропией (теорема Нётер);
    \item Принцип относительности (какой!!?);
    \item Скорость распространения взаимодействий конечна и равна скорости света, причём она одинакова во всех системах отсчёта (!);
    \item Уравнения движения инвариантны относительно преобразований Лоренца.
\end{enumerate}

Пункт №5 представляет наибольший интерес. Приведём вывод преобразований Лоренца.
\begin{figure}[h]
    \centering
    \begin{tikzpicture}
        \draw [->] (0,0) -- (3,0) node [anchor=north west] {$x$};
        \draw [->] (0,0) -- (0,2) node [anchor=south east] {$y$}; 
        \draw [->] (0,0) -- (-1.2,-1)node [anchor=north west] {$z$};
        \draw [thin, dashed] (3,0) -- (5, 0);
        \draw [->] (5,0) -- (8,0) node [anchor=north west] {$x'$};
        \draw [->] (5,0) -- (5,2) node [anchor=south east] {$y'$}; 
        \draw [->] (5,0) -- (3.8,-1)node [anchor=north west] {$z'$};
    
        \node [anchor = south west] at (0, 0) {$0$};
        \node [anchor = south west] at (5, 0) {$0'$};
        \node [] at (1,1.5) {\Large $K$};
        \node [] at (6,1.5) {\Large $K'$};
    
        \draw [->, thick] (7, 0.6) -- (8, 0.6);
        \node [anchor = south] at (7.5, 0.7) {$\vec{V}$};
    \end{tikzpicture}

    \caption{к выводу преобразований Лоренца}
\end{figure}
\begin{enumerate}
    \item Из однородности пространства-времени следует, что преобразования должны быть линейными. Здесь и далее мы всегда будем считать, что движение происходит только вдоль оси $X$,
    соответственно, преобразуется только ось $X$, а две другие остаются неизменными. С учётом сказанного, преобразования должны иметь вид
        \begin{numcases}{}
            x' = \alpha x + \beta y + p, \label{x1} \\
            y' = y, \notag \\
            z' = z, \notag \\
            t' = \sigma x + \delta t + q. \label{t2}
        \end{numcases}

    \item Константы $p$ и $q$ положим равными нулю, так как они связаны с выбором начальных условий, то есть $x_0 = p,\: t_0 = q$.
    
    \item Пусть $V$ --- скорость системы отсчёта $K'$ относительно $K$-системы. Тогда $\alpha = \alpha(V), \beta = \beta(V), \sigma = \sigma(V), \delta = \delta(V).$
    Также имеем
    \[
        x' = 0 \leftrightarrow x = V \cdot t
    \] --- это по определению того, что начало отсчёта $0'$ движется вдоль оси $X$ со скоростью $V$ в системе $K$. Подставим это в уравнение (\ref{x1}):
    \[
        0 = \alpha V t + \beta t \Rightarrow \beta = - \alpha V.
    \]
    Тогда уравнения (\ref{x1}) и (\ref{t2}) примут вид:
    \[ \begin{cases}
        x' = \alpha(V)\brackets{x - V t}; \\
        t' = \sigma(V)x + \delta(V) \tag{\Romannum{1}}\label{strI}
    \end{cases} \]
    Сразу сделаем оговорку об {\it обратных} преобразованиях Лоренца, которые получаются из {\it прямых} преобразований заменой $V$ на $-V$:
    \[ \begin{cases}
        x = \alpha(-V)\brackets{x' + V t'}; \\
        t = \sigma(-V)x' + \delta(-V) t' \tag{\Romannum{2}}\label{revII}
    \end{cases} \]

    \item Подставим формулы (\ref{strI}) в систему (\ref{revII}):
    \begin{multline*}
        x = \alpha(-V) \Bigl[\alpha(V) \brackets{x - V t} + V \brackets{\sigma(V) + \delta(V) t}\Bigr] = \\
        = x\Bigl[\alpha(-V)\alpha(V) + \alpha(-V)V \cdot \sigma(V)\Bigr] + t \cdot \alpha(-V) \Bigl[V\delta(V) - V\alpha(V)\Bigr].
    \end{multline*}
    В силу того, что написанное выше уравнение верно для всех $t$, множитель при $t$ полагаем равным нулю, откуда следует:
    \begin{gather*}
        \delta(V) = \alpha(V); \\
        \alpha(-V)\alpha(V) + \alpha(-V)\sigma(V) = 1.
    \end{gather*}

    \item Из изотропии пространства следует, что $\alpha(V) = \alpha(-V)$, и поэтому
    \begin{equation}
        \alpha^2(V) + \alpha(V) \sigma(V) = 1 \label{alpha3}
    \end{equation}

    \item Пусть из точки $0$ в момент совпадения точек $0$ и $0'$ испущен световой сигнал, направленный по оси $X$. Из свойства №4 следует
    \[
        \frac{x}{t} = \frac{x'}{t'} = c 
        = \frac{\alpha\brackets{x - V t}}{\sigma x + \alpha t} = \frac{\alpha\brackets{\frac{x}{t} - V}}{\sigma \frac{x}{t} + \alpha}
        = \overset{\textrm{\it замена}}{\left| \frac{x}{t} = c \right|} = \frac{\alpha \brackets{c - V}}{\sigma c + \alpha},
    \]
    откуда следует
    \[
        \sigma c^2 + \alpha c = \alpha c - \alpha V \Rightarrow \sigma = -\frac{V}{c^2}.
    \]
    Подстановкой последнего равенства в (\ref{alpha3}) получаем
    \[
        \alpha^2 - \frac{V^2}{c^2} \alpha^2 = 1 \Rightarrow \alpha(V) = \frac{1}{\sqrt{1 - V^2 \big/ c^2}}.
    \]
    \begin{note}
        Формально, $\displaystyle \alpha(V) = \frac{\pm 1}{\sqrt{1 - V^2 \big/ c^2}}$. Случай со знаком <<минус>> называется {\it несобственным преобразованием Лоренца}, при котором производится инверсия по координатам.
    \end{note}
\end{enumerate}

Мы получили всё необходимое, чтобы выписать {\it преобразования Лоренца}.

    \[
        \begin{cases} 
            \displaystyle x' = \frac{x - V t}{\sqrt{1 - V^2 \big/ c^2}} \\
            \displaystyle y' = y \\
            \displaystyle z' = z \\
            \displaystyle t' = \frac{t - \frac{V}{c^2} \cdot x}{\sqrt{1 - V^2 \big/ c^2}}
        \end{cases}
        \longleftrightarrow \hspace{0.8cm}
        \begin{cases}
            \displaystyle x = \frac{x' + V t'}{\sqrt{1 - V^2 \big/ c^2}} \\
            \displaystyle y = y' \\
            \displaystyle z = z' \\
            \displaystyle t = \frac{t' + \frac{V}{c^2} \cdot x'}{\sqrt{1 - V^2 \big/ c^2}}
        \end{cases}
    \]

    Для простоты вводят обозначения $\displaystyle \beta = \frac{\beta}{c}, \ \gamma = \frac{1}{\sqrt{1 - V^2 \big/ c^2}}$ --- \textit{относительная скорость} и \textit{релятивистский фактор} соответственно. Преобразования примут вид (выписаны только нетривиальные уравнения):
    \[
        \begin{cases}
            x' = \gamma\brackets{x - \beta c t} \\
            ct' = \gamma\brackets{ct - \beta x}
        \end{cases}
        \longleftrightarrow \hspace{0.8 cm}
        \begin{cases}
            x = \gamma\brackets{x' + \beta c t'} \\
            ct = \gamma\brackets{ct' + \beta x'}
        \end{cases}
    \]
    Рассмотрим также матричную запись преобразований Лоренца:
    \[
    \begin{aligned}
        \begin{bmatrix} ct' \\ x' \\ y' \\ z' \end{bmatrix}
            = &
        \begin{bmatrix}
            \gamma & -\gamma \beta & 0 & 0 \\
            -\gamma \beta & \gamma & 0 & 0 \\
            0 & 0 & 1 & 0 \\
            0 & 0 & 0 & 1
        \end{bmatrix}
        \cdot 
        \begin{bmatrix} ct \\ x \\ y \\ z \end{bmatrix}
        \textrm{--- прямое преобразование} \\
        \begin{bmatrix} ct \\ x \\ y \\ z \end{bmatrix}
            = &
        \begin{bmatrix}
            \gamma & \gamma \beta & 0 & 0 \\
            \gamma \beta & \gamma & 0 & 0 \\
            0 & 0 & 1 & 0 \\
            0 & 0 & 0 & 1
        \end{bmatrix}
        \cdot 
        \begin{bmatrix} ct' \\ x' \\ y' \\ z' \end{bmatrix}
        \textrm{--- обратное преобразование}
    \end{aligned}
    \]

\subsection{Следствия преобразований Лоренца}
\begin{enumerate}
    \item {\bf Сокращение длин.}
        Рассмотрим две точки $x_1$ и $x_2$, положения которых были зафиксированы в один и тот же момент времени: $t_1 = t_2$. Расстояние между точками в неподвижной системе обозначим как $l_0 = x_1 - x_2$. Координаты этих точек в $K'$"--~системе равны соответственно
        \begin{gather*}
            x_1' = \frac{x_1 - V t_1}{\sqrt{1 - V^2 \big/c^2 }}, \\
            x_2' = \frac{x_2 - V t_2}{\sqrt{1 - V^2 \big/c^2 }}.
        \end{gather*} 
        Расстояние между этими точками в $K'$"--~системе равно
        \[
            l = x_1' - x_2' = \frac{x_1 - x_2}{\sqrt{1 - V^2 \big/ c^2}} \Rightarrow l = l_0 \sqrt{1 - V^2 \big/ c^2}.
        \]
        \begin{note}
            Так как координаты $y$ и $z$ остаются неизменными, объёмы и площади преобразуются так же:
            \[ \begin{aligned}
                \mathcal{V} = \mathcal{V}_0 \sqrt{1 - V^2 \big/ c^2}, \\
                S = S_0 \sqrt{1 - V^2 \big/ c^2}.
            \end{aligned} \]
        \end{note}

    \item {\bf Относительность одновременности.} Пусть в точках $x_1'$ и $x_2'$ произошли события, одновременные в $K'$"--~системе: $t_1' = t_2'$. Но в $K$"--~системе
        \begin{gather*}
            t_1 = \frac{t_1' + \frac{V}{c^2} x_1'}{\sqrt{1 - V^2 \big/ c^2}}, \
            t_2 = \frac{t_2' + \frac{V}{c^2} x_2'}{\sqrt{1 - V^2 \big/ c^2}}, \\
            t_1 - t_2 = \frac{V}{c^2} \cdot \frac{x_1' - x_2'}{\sqrt{1 - V^2 \big/ c^2}}. 
        \end{gather*}
        Таким образом, эти события одновременны и в $K$"--~ системе только в том случае, если $x_1' = x_2'$.

    \item {\bf Замедление времени.} Пусть в точке $x_1'$ имеются часы, которые покоятся в системе $K'$ ($x_1' = const$). Отмерим этими часами промежуток времени от $t_1'$ до $t_2'$:
        \[
            x_2' = x_1' \Rightarrow t_1 - t_2 = \frac{t_1' - t_2'}{\Relroot}.
        \]
        Величина $\Delta t' = t_1' - t_2' = \Delta \tau$ называется {\it собственным временем}, а величина $\Delta t = t_1 - t_2$ --- {\it лабораторным временем}:
        \[
            \Delta t = \frac{\Delta \tau}{\Relroot}, \ \Delta \tau = \Delta t \cdot \Relroot.
        \]
        \begin{note}
            Для неравномерного движения формулы можно записать в интегральной форме, например
            \[
                \Delta \tau = \int\limits_{t_1}^{t_2} dt \sqrt{1 - \frac{V^2(t)}{c^2}}.
            \]
        \end{note}
\end{enumerate}

\subsection{Закон сложения скоростей.}
    Пусть некая частица движется со скоростью $\vec{v} = \frac{\vec{dr}}{dt}$ в системе $K$, а в системе $K'$ её скорость равна $\vec{v'} = \frac{\vec{dr'}}{dt'}$. Запишем дифференциалы преобразований Лоренца:
    \begin{gather*}
        dx' = \frac{dx - V dt}{\Relroot}; \\
        dy' = dy; \\
        dz' = dz; \\
        dt' = \frac{dt - \frac{V}{c^2} \cdot dx}{\Relroot}.
    \end{gather*}
    Тогда:
    \begin{gather*}
        v_x' = \frac{dx'}{dt'} = \frac{dx - V dt}{dt - \frac{V}{c^2} dx} = \frac{\frac{dx}{dt} - V}{1 - \frac{V}{c^2}\cdot\frac{dx}{dt}} = \frac{v_x - V}{1 - \frac{V v_x}{c^2}}; \\
        v_y' = \frac{dy'}{dt'} = \frac{dy \cdot \Relroot}{dx - \frac{V}{c^2}dt} = v_y \cdot \frac{\Relroot}{1 - \frac{V v_x}{c^2}}; \\
        v_z' = v_z \cdot \frac{\Relroot}{1 - \frac{V v_x}{c^2}}.
    \end{gather*}
    Обратные преобразования получаются заменой $V$ на $-V$:
    \begin{gather*}
        v_x = \frac{v_x' + V}{1 + \frac{V v_x'}{c^2}}; \\
        v_y = v_y' \cdot \frac{\Relroot}{1 + \frac{V v_x'}{c^2}}; \\
        v_z' = v_z' \cdot \frac{\Relroot}{1 + \frac{V v_x'}{c^2}}.
    \end{gather*}

\subsection{Преобразования углов.}
    Рассмотрим частицу, которая движется со скоростью $\vec{v}$ в $K$-системе. Угол $\theta$ между направлением оси $x$ и скоростью назовём
    \textit{полярным}, а угол $\phee$ между осью $y$ и проекцией скорости на плоскость $z0y$ --- \textit{азимутальным}.

    \begin{figure}[h]
        \centering
        \begin{tikzpicture}
    \draw [->] (0,0) -- (3,0) node [anchor=north west] {$x$};
    \draw [->] (0,0) -- (0,2) node [anchor=south east] {$y$}; 
    \draw [->] (0,0) -- (-1.2,-1)node [anchor=north west] {$z$};
    \draw [->] (5,0) -- (8,0) node [anchor=north west] {$x'$};
    \draw [->] (5,0) -- (5,2) node [anchor=south east] {$y'$}; 
    \draw [->] (5,0) -- (3.8,-1)node [anchor=north west] {$z'$};

    \node [] at (1,1.8) {\Large $K$};
    \node [] at (6,1.8) {\Large $K'$};
    \draw [->, thick] (7.5, 1.1) -- (8.5, 1.1);
    \node [anchor = south] at (8, 1.2) {$\vec{V}$};

    \draw [->, thick] (0,0) -- (1.5, 1) node [anchor = south west] {$\vec{v}$};
    \draw [->, thick] (5,0) -- (6.7, 0.6) node [anchor = south west] {$\vec{v'}$};

    \coordinate (a1) at (1,0);
    \coordinate (a2) at (0.832,0.555);
    \coordinate (o) at (0,0);
    \pic [draw] {angle = a1--o--a2};
    \node [] at (0.7, 0.2) {$\theta$};

    \coordinate (b1) at (7,0);
    \coordinate (b2) at (6.886,0.665);
    \coordinate (q) at (5,0);
    \pic [draw] {angle = b1--q--b2};
    \node [] at (6.1, 0.2) {$\theta'$};
    
    \filldraw [black] (-1,1) circle (1pt);
    \draw [thin, dashed] (-1, 1) -- (1.5,1);
    \draw [thin, dashed] (-1, 1) -- (-1,-0.833);
    \draw [thin, dashed] (-1, 1) -- (0,1.833);
    \draw [thin, dashed, ->] (0,0) -- (-1, 1);
    \coordinate (f1) at (0, 1);
    \coordinate (f2) at (-0.7, 0.7);
    \pic [draw] {angle = f1--o--f2};
    \node [] at (-0.2, 0.7) {$\varphi$};

    
    \filldraw [black] (4,0.6) circle (1pt);
    \draw [thin, dashed] (4, 0.6) -- (6.7, 0.6);
    \draw [thin, dashed] (4, 0.6) -- (5, 1.433);
    \draw [thin, dashed] (4, 0.6) -- (4, -0.833);
    \draw [thin, dashed] (4, 0.6) -- (5, 0);
    \coordinate (g1) at (5,1);
    \coordinate (g2) at (4, 0.6);
    \pic [draw] {angle = g1--q--g2};
    \node [] at (4.7,0.7) {$\varphi'$};
\end{tikzpicture}

        \caption{Преобразование углов}
    \end{figure}

    Азимутальный угол не преобразуется:
    \[
        \tg \phee ' = \frac{v_y'}{v_z'} = \frac{v_y}{v_z} = \tg \phee \Rightarrow \phee' = \phee.
    \]
    Полярный угол $\theta$ преобразуется согласно следующим формулам:
    \[
        \tg \theta = \frac{\sqrt{v_y^2 + v_z^2}}{v_x} = \frac{\sqrt{v_y^{'2} + v_z^{'2}} \cdot \Relroot}{v_x' + V}.
    \]
    Используя равенства $v_x' = v' \cos\theta'$ и $v' \sin \theta' = \sqrt{v_y^{'2} + v_z^{'2}}$, получаем
    \[
        \tg \theta = \frac{v' \sin \theta' \cdot \Relroot}{v' \cos \theta' + V} = \frac{v' \sin \theta'}{\gamma \brackets{v' \cos \theta' + V}}.
    \]
    Применительно к пучкам света ($v' = c$) получаем
    \[
        \tg \theta = \frac{\sin \theta'}{\gamma \brackets{\cos\theta' + \beta}}.
    \]
    Используя формулы преобразования $x$-компоненты скорости ($v_x = c \cos \theta, \ v_x' = c \cos \theta'$), можно получить ещё одну форму записи:
    \begin{gather*}
        \cos \theta' = \frac{\cos \theta - \frac{V}{c}}{1 - \frac{V}{c}\cos\theta}; \\
        \cos \theta = \frac{\cos \theta' + \frac{V}{c}}{1 + \frac{V}{c}\cos\theta'}.
    \end{gather*}

    Пусть $V \ll c$, тогда
    \[
        \cos \theta' = \cos \theta = \cos \theta' - \frac{\cos \theta' + \frac{V}{c}}{1 + \frac{V}{c} \cos \theta'} =
        \frac{\frac{V}{c} \brackets{\cos^2\theta' - 1}}{1 + \frac{V}{c}\cos\theta'} \approx - \frac{V}{c} \sin^2\theta' \approx -\brackets{\theta' - \theta} \sin \theta.
    \]
    На предпоследнем шаге мы пренебрегли знаменателем ввиду условия $V \ll c$, а на последнем применили формулу Лагранжа и учли, что $\theta \approx \theta'$ из-за малости скорости.
    Окончательный результат:
    \[
        \boxed{\theta' - \theta = \Delta \theta \approx \frac{V}{c} \sin \theta}
    \]

\subsection{Интервал. Пространство Минковского.}
    Введём обозначения:
    \[
        x^0 = ct, \ x^1 = x, \ x^2 = y, \ x^3 = z.
    \]
    Преобразования Лоренца в этих обозначениях выглядят так:
    \[ \begin{cases}
        x^{0'} = \gamma \brackets{x^0 - \beta x^1} \\
        x^{1'} = \gamma \brackets{x^1 - \beta x^0} \\
        x^{2'} = x^2 \\
        x^{3'} = x^3
    \end{cases} \]
    Введём также величину $\psi : \ch \psi = \gamma = \frac{1}{\Relroot}$, причём
    \[
        \sh \psi = \sqrt{\ch^2 \psi - 1} = \sqrt{\frac{1}{1 - \beta^2} - 1} = \frac{\beta}{\sqrt{1 - \beta^2}} = \beta \gamma.
    \]
    Запишем преобразования Лоренца уже в этих обозначениях:
    \[ \begin{cases}
        x^{0'} = x^0 \ch \psi - x^1 \sh \psi \\
        x^{1'} = - x^0 \sh \psi + x^1 \ch \psi \tag{\Romannum{3}}\label{hyp}
    \end{cases} \]
    Преобразования вида (\ref{hyp}) называют {\it гиперболическим поворотом}. Очевидна аналогия с обычным ортогональным преобразованием поворота:
    \[ \begin{cases}
        x' = x\cos\phee + y\sin\phee \\
        y' = -x\sin\phee + y\cos\phee
    \end{cases} \]
    Как обычный поворот сохраняет неизменными расстояния ($r = inv$), так и гиперболический поворот имеет инвариант, который называется {\it интервалом}. Найдём его. 
    Пусть $\psi = i\phee$, и тогда $\ch\psi = \ch i\phee = \cos\phee$, $\sh\psi = \sh i\phee = i\sin\phee$. Преобразования примут вид, похожий на ортогональный поворот:

    \[ \begin{cases}
        x^{0'} = x^0 \cos\phee - ix^1\sin\phee \\
        x^{1'} = -ix^0\sin\phee + x^1\cos\phee \ \leftrightarrow \ ix^{0'} = ix^0\cos\phee + x^1\sin\phee
    \end{cases} \]
    По аналогии с инвариантом ортогонального поворота:
    \[
        r_{12}^2 = \brackets{x_2 - x_1}^2 + \brackets{y_2 - y_1}^2 = \brackets{x_2' - x_1'}^2 + \brackets{y_2' - y_1'}^2,
    \]
    введём величину $s$:
    \[
        -s_{12}^2 = \brackets{ix_2^0 - ix_1^0}^2 + \brackets{x_2^1 - x_1^1}^2 = -c^2\brackets{t_2 - t_1}^2 + \brackets{x_2 - x_1}^2,
    \]
    которую назовём {\it интервалом}. Обобщая это определение на все координаты,
    \[
        \boxed{s_{12}^2 =c^2\brackets{t_2 - t_1}^2 - \brackets{x_2 - x_1}^2 - \brackets{y_2 - y_1}^2 - \brackets{z_2 - z_1}^2 = c^2 \brackets{t_2 - t_1}^2 - r_{12}^2}
    \]

    \begin{note}
        $s_{12} = 0$ не означает совпадения событий.
    \end{note}

    Различают:
    \begin{itemize}
        \item $s_{12}^2 > 0$ --- временноподобный интервал, для которого $\exists K' : s_{12}^2 = c^2\brackets{t_2' - t_1'}^2, x_1' = x_2'$;
        \item $s_{12}^2 < 0$ --- пространственноподобный интервал, для которого $\exists K' : s_{12}^2 = -r_{12}^2, t_1' = t_2'$;
        \item $s_{12}^2 = 0$ --- светоподобный интервал.
    \end{itemize}

    \begin{figure}[h]
        \centering
        \begin{tikzpicture}
\draw [thick, ->] (-4.5, 0) -- (4.6, 0) node [anchor = north west] {\Large $ct$};
\draw [thick, ->] (0, -4.5) -- (0, 4.6) node [anchor = south west] {\Large $x$};
\draw [thin] (0,0) -- (4, 4) node [midway, above, sloped] {$s^2 = 0$};
\draw [thin] (0,0) -- (4, -4);
\draw [thin] (0,0) -- (-4, -4);
\draw [thin] (-4, 4) -- (0,0) node [midway, above, sloped] {$s^2 = 0$};

\node [] at (2, 0.3) {$s^2 > 0$};
\node [] at (-2, -0.3) {$s^2 > 0$};
\node [] at (1, 3) {$s^2 < 0$};
\node [] at (-1, -3) {$s^2 < 0$};
\end{tikzpicture}

        \caption{<<Cветовой конус>> Минковского}
    \end{figure}
    \newpage
    $s_{12}^2 > 0$ отвечает наличию причинно"--~следственной связи между событиями: в любой системе отсчта можно сказать, какое из них произошло раньше.
