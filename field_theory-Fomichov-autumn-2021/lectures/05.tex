\newpage
\section{Тензор электромагнитного поля}
\subsection{Понятие о тензоре}
    Рассмотрим два 4-вектора $A^i$ и $B^i$, преобразование Лоренца с матрицей 
    ${\alpha^i}_k : \brackets{A^i}' = {\alpha^i}_k A^k, \brackets{B^i}' = {\alpha^i}_kB^k$.
    \textit{Прямое произведение} 4-векторов есть матрица вида
    \[
        A^iB^k = \begin{bmatrix}
            A^0B^0 & \dots & A^0B^3 \\
            \vdots & \ddots & \vdots \\
            A^3B^0 & \dots & A^3B^3
        \end{bmatrix}
    \]
    Прямое произведение преобразованных векторов равно
    \[
        \brackets{A^iB^k}' = \brackets{A^i}'\brackets{B^k}' = {\alpha^i}_kA^m{\alpha^k}_nB^n = 
        {\alpha^i}_m{\alpha^k}_nA^mB^n.
    \]
    Пусть теперь мы имеем дело с произвольной двухиндексной (типа матрицы) величиной $T^{ik}$.
    \begin{Def} Если величина $T^{ik}$  при смене системы отсчёта преобразуется по правилу
        \[
            \brackets{T^{ik}}' = {\alpha^i}_m{\alpha^k}_n T^{mn},
        \]
        то $T^{ik}$ называется \textit{контрвариантным тензором второго ранга}.
    \end{Def}
    Очевидным образом определени обобщается на тензоры высших рангов, например, $T^{ikl}$ --- тензор третьего ранга, если
    $\brackets{T^{ikl}}' = {\alpha^i}_k{\alpha^k}_n {\alpha^l}_p T^{mnp}$.
    Тензор может быть записан и в ковариантном виде:
    \[
        T'_{ik} = {\alpha_i}^m {\alpha_k}^nT_{mn}
    \]
    а также в смешанной форме:
    \[
        \brackets{{T_i}^k}' = {\alpha_i}^m{\alpha^k}_n {T_m}^n
    \]
    Компоненты контрвариантной, ковариантной и смешанной форм связаны между собой так же, как компоненты прямого произведения двух 4-векторов
    в контрвариантной, ковариантной или двух разных формах записи соответственно. Например,
    \[
        A_0 = A^0,\: A_{\alpha} = -A^{\alpha} \: \Rightarrow \: T_{00} = T^{00},\: T_{0\alpha} = -T^{0\alpha},\: T_{\alpha\beta} = T^{\alpha\beta}.
    \]
    \begin{example}
        Рассмотрим 4-вектор градиента:
        \[
            \nabla_i = \pard{}{x^i} = \brackets{\frac{1}{c}\pard{}{t}, \: \vec{\nabla}}
        \]
        Свёртки 4-вектора градиента с 4-вектором  $A^i$ дают контрвариантный, ковариантный и смешанные тензоры второго ранга 
        $\nabla_iA_i, \: \nabla_iA^i,\: \nabla^iA_i,\: \nabla^iA^i$.
    \end{example}
    Существуют два инвариантных тензора второго ранга:
    \begin{enumerate}
        \item Метрический тензор:
            \[
                g_{ik} = \begin{bmatrix}
                    1 & 0 & 0 & 0  \\
                    0 & -1 & 0 & 0\\
                    0 & 0 & -1 & 0 \\
                    0 & 0 & 0 & -1
                \end{bmatrix} = g^{ik}, \:\:
                {g_i}^k = \begin{bmatrix}
                    1 & 0 & 0 & 0  \\
                    0 & 1 & 0 & 0  \\
                    0 & 0 & 1 & 0  \\
                    0 & 0 & 0 & 1 
                \end{bmatrix} = \delta_i^k.
            \]
            Проверим инвариантность метрического тензора:
            \[
                \brackets{{g_i}^k}' = {\alpha_i}^m{\alpha^k}_n {g_m}^n = {\alpha_i}^m{\alpha^k}_n\delta_m^n =
                {\alpha_i}^m{\alpha^k}_m = \delta_i^k = {g_i}^k.
            \]
        \item Абсолютно антисимметричный тензор $e^{iklm}$ --- аналог символа Леви-Чивиты для четырёхмерного пространства.
            \[
                e^{iklm} = \begin{cases}
                    \pm 1, \:\: i \not= k \not= l \not=m, \: i \not= l, \: k \not= m \\
                    0 \: \textrm{ в остальных случаях}
                \end{cases}
            \]
            
                Как и в трёхмерном случае, $e^{0123} = 1, \: e^{1023} = -1$ а циклические перестановки индексов не меняют знак. Всего
                тензор содержит $4! = 24$ ненулевых компоненты. Покажем, что абсолютно антисимметричный тензор является инвариантом:
                \[
                    \brackets{e^{iklm}}' = {\alpha^i}_p {\alpha^k}_q {\alpha^l}_r {\alpha^m}_s e^{pqrs}.
                \]
                Рассмотрим один элемент преобразованного тензора, например, $e^{0123}$:
                \[
                    \brackets{e^{0123}}' = {\alpha^0}_p {\alpha^1}_q {\alpha^2}_r {\alpha^3}_s e^{pqrs},
                \]
                а это выражение в точности представляет собой определитель матрицы преобразования Лоренца ${\alpha^i}_k$:
                \[
                    \brackets{e^{0123}}' = \det\begin{Vmatrix}
                        {\alpha^i}_k
                    \end{Vmatrix} = \det\begin{Vmatrix}
                        \ch\psi & -\sh\psi & 0 & 0\\
                        -\sh\psi & \ch\psi & 0 & 0 \\
                        0 & 0 & 1 & 0 \\
                        0 & 0 & 0 & 1
                    \end{Vmatrix} = 1 = e^{0123}
                \]
                При перестановке индексов мы переставляем строки в матрице под знаком определителя. Как известно из курса линейной алгебры, при перестановке строк
                матрицы её определитель меняется точно так же, как при перестановках индексов тензора $e^{iklm}$, то есть, циклические перестановки сохраняют знак, а прочие --- нет.
                Значит, для любых $i, k, l, m$ верно
                \[
                    \brackets{e^{iklm}}' = e^{iklm}.
                \]
                \begin{note}
                    Можно ввести ковариантный тензор $e_{iklm}$. В случае с произвольными тензорами действует следующее правило: каждое <<опускание>> индекса, не равного нулю,
                    меняет знак тензора. В случае тензора $e^{iklm}$ все ненулевые элементы имеют ровно один ненулевой индекс, поэтому $e_{iklm} = -e^{iklm}$.
                \end{note}
                \begin{note}
                    $e^{0\alpha\beta\gamma} = e_{\alpha\beta\gamma}$ --- при $i=0$ тензор совпадает с трёхмерным символом Леви-Чивиты.
                    Напоминаем, что в трёхмерном пространстве нет различия между ковариантной и контрвариантной формами записи.
                \end{note}
    \end{enumerate}

    Ещё одна инвариантная величина, хотя и не являющаяся тензором --- \textit{четырёхмерный объём}:
    \[
        d^4x = dx^0dx^1dx^2dx^3 = \begin{vmatrix}
            \pard{\brackets{x^0,\ x^1,\ x^2,\ x^3}}{\brackets{x^{0'},\ x^{1'},\ x^{2'},\ x^{3'}}}
        \end{vmatrix} d^4x',
    \]
    где $x^i = {\alpha_k}^i x^{k'}$. Так как $\displaystyle \pard{x^i}{x^{k'}} = {\alpha_k}^i$, то
    \[
        \begin{vmatrix}
            \pard{\brackets{x^0,\ x^1,\ x^2,\ x^3}}{\brackets{x^{0'},\ x^{1'},\ x^{2'},\ x^{3'}}}
        \end{vmatrix} = \det\begin{Vmatrix}
            {\alpha_k}^i
        \end{Vmatrix}
    \]
    (якобиан преобразования равен определителю матрицы преобразования, т.к. преобразования Лоренца линейные).
    \[
        \det\begin{Vmatrix}
            {\alpha_k}^i
        \end{Vmatrix} = 1 \: \Rightarrow \: d^4x = d^4x'
    \]

\subsection{Структура тензора второго ранга}
    Рассмотрим произвольный четырёхмерный тензор второго ранга:
    \[
        F^{ik} = 
        \begin{bmatrix}
            F^{00} & \vdots & F^{01} & F^{02} & F^{03} \\
            \dots & \dots & \dots & \dots & \dots \\
            F^{10} & \vdots & F^{11} & F^{12} & F^{13} \\
            F^{20} & \vdots & F^{21} & F^{22} & F^{23} \\
            F^{30} & \vdots & F^{31} & F^{32} & F^{33} \\
        \end{bmatrix}
    \]
    Величина $F^{00}$ не связана с пространственными координатами и является пространственным скаляром (не изменяется
    при замене координат, но преобразуется при смене системы отсчёта).
    Величины $F^{01}, F^{02}, F^{03} = F_{0x}, F_{0y}, F_{0z}$ образуют пространственный (трёхмерный) вектор, как и величины $F^{10}, F^{20}, F^{30}$.
    Оставшиеся 9 компонент образуют пространственный тензор.

    \begin{Def}
        Тензор \Romannum{2} ранга называют \textit{симметричным}, если для всех $i, k$ верно $F^{ik} = F^{ki}$.
        Тензор \Romannum{2} ранга называют \textit{антисимметричным}, если $F^{ik} = -F^{ki}, \: i \not= k$.
    \end{Def}
    \begin{prop}
        Свойство антисимметричности абсолютно: если тензор антисимметричен  водной системе отсчёта, то он будет антисимметричным и в любой другой
        систее отсчёта.
    \end{prop}

    Антисимметричный тензор \Romannum{2} ранга можно записать в виде
    \[
        F^{ik} = \begin{bmatrix}
            0 & -P_x & -P_y & -P_z \\
            P_x & 0 & -a_z & a_y \\
            P_y & a_z & 0 & -a_x \\
            P_z & -a_y & a_x & 0
        \end{bmatrix} \equiv \begin{Vmatrix}
            \vec{P} & \vec{a}
        \end{Vmatrix}
    \]
    Вектор $P_{\alpha} = F^{\alpha 0}$ называют \textit{полярным}, а вектор $a_{\alpha} = -\frac 12 e_{\alpha\beta\gamma} F^{\beta\gamma}$ --- \textit{аксиальным}.
    Ковариантный тензор, очевидно, имеет вид
    \[
        F_{ik} = \begin{bmatrix}
            0 & P_x & P_y & P_z \\
            -P_x & 0 & -a_z & a_y \\
            -P_y & a_z & 0 & -a_x \\
            -P_z & -a_y & a_x & 0
        \end{bmatrix}
    \]
\subsection{Структура тензора электромагнитного поля}
    В случае электромагнитного поля $\vec{P} = \vec{E}$, а $\vec{a} = \vec{H}$:
    \[
        F_{ik} = \begin{bmatrix}
            0 & E_x & E_y & E_z \\
            -E_x & 0 & -H_z & H_y \\
            -E_y & H_z & 0 & -H_x \\
            -E_z & -H_y & H_x & 0 \\
        \end{bmatrix} = \begin{bmatrix}
            0 & \vec{E} \\
            -\vec{E} & -e_{\alpha\beta\gamma}H_{\gamma}
        \end{bmatrix}
    \]
    Рассмотрим вопрос о преобразовании тензора $F_{ik}$ при смене системы отсчёта. Для 4-вектора верно:
    \begin{gather*}
        \brackets{A^0}' = \gamma\brackets{A^0 - \beta A^1}, \:
        \brackets{A^1}' = \gamma\brackets{A^1 - \beta A^0} \\
        \brackets{A^2}' = A^2, \:
        \brackets{A^3}' = A^3, \\
        F^{ik} = \begin{bmatrix}
            0 & -\vec{E} \\
            \vec{E} & -e_{\alpha\beta\gamma}H_{\gamma}
        \end{bmatrix} \cdot F^{\alpha 0} = E_{\alpha}.
    \end{gather*}
    Для примера возьмём элемент $F^{20}$. По индексу $2$ преобразования нет, поэтому этот элемент преобразуется так же, как нулевая компонента 4-вектора:
    \[
        \brackets{F^{20}}' = \gamma\brackets{F^{20} - \beta F^{21}}.
    \]
    Аналогичным образом можно легко написать преобразования следующих элементов:
    \begin{gather*}
        \brackets{F^{30}}' = \gamma\brackets{F^{30} - \beta F^{31}}, \\
        \brackets{F^{21}}' = \gamma\brackets{F^{21} - \beta F^{20}}, \\
        \brackets{F^{31}}' = \gamma\brackets{F^{31} - \beta F^{30}}.
    \end{gather*}
    Эти же преобразования в обозначениях тензора электромагнитного поля:
    \begin{gather*}
        E_y' = \gamma\brackets{E_y - \beta H_z}, \\
        E_z' = \gamma\brackets{E_z + \beta H_y}, \\
        H_y' = \gamma\brackets{H_y + \beta E_z}, \\
        H_z' = \gamma\brackets{H_z - \beta E_y}.
    \end{gather*}
    Так как индексы $2$ и $3$ не преобразуются, $\brackets{F^{23}}' = F^{23}$, откуда $H_z' = H_z$. Наконец, элемент $F^{01}$ преобразуется
    сразу по двум индексам:
    \begin{gather*}
        \brackets{F^{01}}' = F^{0'1'} = \gamma\brackets{F^{0'1} -\beta F^{0'0}} =
        \gamma\brackets{\gamma\brackets{F^{01} - \beta F^{11}} - \beta\gamma\brackets{F^{00} - \beta F^{10}}} = \\
        = \gamma^2\brackets{F^{01} + \beta^2F^{10}} = \gamma^2\brackets{1 - \beta^2}F^{01} = F^{01} \: \Rightarrow \: E_x' = E_x.
    \end{gather*}
    Полученные законы преобразования полей запишем в инвариантной по отношению к вектору $\vec{V}$ скорости $K'$-системы в $K$-системе форме:
    \[
        \boxed{
            \begin{matrix}
                \displaystyle \vec{E_{\parallel}}' = \vec{E_{\parallel}} \\
                \displaystyle \vec{H_{\parallel}}' = \vec{H_{\parallel}} \\
                \displaystyle \vec{E_{\perp}}' = \gamma\brackets{\vec{E_{\perp}} + \frac 1c \sqbrk{\vec{V}\times\vec{H}}} \\
                \displaystyle \vec{H_{\perp}}' = \gamma\brackets{\vec{H_{\perp}} - \frac 1c \sqbrk{\vec{V}\times\vec{E}}}
            \end{matrix}
        }
    \]
\subsection{Инварианты электромагнитного поля}
    Изучим вопрос инвариантных комбинаций полей $\vec{E}$ и $\vec{H}$. Попытаемся найти линеные инварианты:
    \[
        F^{ik} = \begin{bmatrix}
            0 & -\vec{E} \\
            \vec{E} & -e_{\alpha\beta\gamma}H_{\gamma}
        \end{bmatrix} \: \Rightarrow \: {F_i}^i = g_{ik}F^{ik} = 0
    \]
    Это значит, что линейных инвариантов нет. Зато есть квадратичные:
        \begin{gather*}
            F_{ik}F^{ik} = -2\brackets{\vec{E}}^2 + e_{\alpha\beta\gamma}H_{\gamma}\cdot e_{\alpha\beta\delta}H_{\delta} = 
            -2\brackets{\vec{E}}^2 + \brackets{e_{\alpha\beta\gamma} e_{\alpha\beta\delta}}H_{\gamma}H_{\delta} = \\
            = -2\brackets{\vec{E}}^2 + 2\delta_{\gamma}^{\delta}H_{\gamma}H_{\delta} = -2\brackets{\vec{E}}^2 + 2H_{\delta}H_{\delta}
            = 2\brackets{H^2 - E^2}.
        \end{gather*}
        \begin{gather*}
            e^{iklm}F_{ik}F_{lm} =
            e^{0\alpha\beta\gamma}F_{0\alpha}F_{\beta\gamma} +
            e^{\alpha 0\beta\gamma}F_{\alpha 0}F_{\beta\gamma} +
            e^{\alpha\beta 0\gamma}F_{\alpha\beta}F_{0\gamma} +
            e^{\alpha\beta\gamma 0}F_{\alpha\beta}F_{\gamma 0} = \\
            = 4e^{0\alpha\beta\gamma}F_{0\alpha}F_{\beta\gamma} =
            4e_{\alpha\beta\gamma} \brackets{-E_{\alpha}}\brackets{-e_{\beta\gamma\nu}H_{\nu}} =
            4e_{\alpha\beta\gamma}E_{\alpha}e_{\beta\gamma\nu}H_{\nu} = 4E_{\alpha}e_{\alpha\beta\gamma}e_{\nu\beta\gamma}H_{\nu} = \\
            = 8E_{\alpha}\delta_{\alpha}^{\nu}H_{\nu} = 8E_{\alpha}H_{\alpha} = 8\brackets{\vec{E}\cdot\vec{H}}.
        \end{gather*}

    Второй инвариант можно выразить через так называемый \textit{дуальный тензор} электромагнитного поля
    $\tilde{F}_{ik} = \frac 12 e_{iklm}F^{lm}$:
    \[
        \tilde{F}_{ik} = \begin{bmatrix}
            0 & \vec{H} \\
            -\vec{H} & e_{\alpha\beta\gamma}E_{\gamma}
        \end{bmatrix} = \begin{bmatrix}
            0 & H_x & H_y & H_z \\
            -H_x & 0 & E_z & -E_y \\
            -H_y & -E_z & 0 & E_x \\
            -H_z & E_y & -E_x & 0
        \end{bmatrix}, \:\: \tilde{F}^{ik} = \begin{bmatrix}
            0 & -\vec{H} \\
            \vec{H} & e_{\alpha\beta\gamma}E_{\gamma}
        \end{bmatrix}
    \]
    Тогда $\tilde{F}_{ik}F^{ik} = -4\brackets{\vec{E}\cdot\vec{H}} = inv$
    
    Перечислим основные следствия из наличия этих инвариантов:
    \begin{itemize}
        \item Если в какой-либо системе отсчёта $\abs{\vec{H}} = \abs{\vec{E}}$, то это верно для любой другой системы отсчёта.
        \item Аналогично, неравенства $\abs{\vec{H}} > \abs{\vec{E}}, \: \abs{\vec{H}} < \abs{\vec{E}}$ сохраняются во
            всех системах отсчёта.
        \item Если в какой-либо системе $\vec{H} \perp \vec{E}$, $\angle\brackets{\vec{H}, \vec{E}} > 90\degree$ или $\angle\brackets{\vec{H}, \vec{E}} < 90\degree$,
            то эти утверждения верны во всех системах отсчёта.
        \item Если $\brackets{\vec{E}\cdot\vec{H}} \not= 0$, то $\exists K' : \brackets{\vec{E}} \parallel \brackets{\vec{H}}'$.
        \item Если $\brackets{\vec{E}\cdot\vec{H}} = 0$, то
            \begin{enumerate}
                \item $ H^2 - E^2 > 0 \: \rightarrow \: \exists K' : \brackets{\vec{E}}' = \vec{0} $;
                \item $ H^2 - E^2 < 0 \: \rightarrow \: \exists K' : \brackets{\vec{H}}' = \vec{0} $
            \end{enumerate}
    \end{itemize}
    \begin{example}
        Для случая (1) найдём скорость такой $K'$-системы. Выберем направление движения так, чтобы $\vec{V} \perp \vec{H}, \: \vec{V} \perp \vec{E}$ (см. рисунок).
        \begin{figure} [h]
            \centering
            \begin{tikzpicture}
    \draw [->] (0, 0) -- (5, 0) node [anchor = south west] {$X$};
    \draw [->] (0, 0) -- (0, 2) node [anchor = west] {$\vec{H}$};
    \draw [->] (0, 0) -- (-1.5, -1.1) node [anchor = north west] {$\vec{E}$};
    \draw [->, thick] (1, 1.4) -- (3, 1.4) node [midway, above] {$\vec{V}$};
\end{tikzpicture}
        \end{figure}
        \begin{gather*}
           \brackets{\vec{E}}' = \gamma\brackets{\vec{E} + \frac 1c \sqbrk{\vec{V}\times\vec{H}}}; \\
           \brackets{\vec{E}}' = 0 \: \Rightarrow \: \vec{E} + \frac 1c \vecp{V}{H} = 0. \\
        \end{gather*}
        Домножим полученное уравнение векторно на $\vec{H}$:
        \[
           \vecp{H}{E} + \frac 1c \sqbrk{\vec{H}\times\vecp{V}{H}} = \vecp{H}{E} + \frac{\vec{V}}{c^2}H^2 - \frac{\vec{H}}{c}\dotp{V}{H}.
        \]
        Учитывая, что $\vec{V} \perp \vec{H}$, получаем:
        \[
            \vec{V} = -\frac{c}{H^2}\vecp{H}{E} = \frac{c}{H^2}\vecp{E}{H}.
        \]
        Проделаем аналогичные рассуждения для случая (2):
        \begin{gather*}
            \brackets{\vec{H}}' = \gamma\brackets{\vec{H} - \frac{1}{c}\vecp{V}{E}} = 0; \\
            \vecp{E}{H} - \frac 1c \sqbrk{\vec{E}\times\vecp{V}{E}} = \vecp{E}{H} - \frac{\vec{V}}{c}E^2 = 0; \\
            \vec{V} = \frac{c\vecp{E}{H}}{E^2}.
        \end{gather*}
    \end{example}

