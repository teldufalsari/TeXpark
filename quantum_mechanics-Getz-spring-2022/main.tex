

\documentclass[a4paper,12pt]{article}

\usepackage{cmap}					% поиск в PDF
\usepackage[T2A]{fontenc}			% кодировка
\usepackage[utf8]{inputenc}			% кодировка исходного текста
\usepackage[english,russian]{babel}	% локализация и переносы

\author{Made by timattt}
\title{Лекции по квантовой механике}
\setlength{\leftskip}{-7em}
\setlength{\rightskip}{-7em}

\begin{document} % Конец преамбулы, начало текста.

\maketitle
\newpage

\subsubsection*{Волновая функция} Это комплекснозначная функция, вида $ \Psi = \Psi(\vec r, t) $

\subsubsection*{Плотность верояности}

Квадрат модуля $\Psi$-функции есть вероятность обнаружить частицу в заданной точке в заданное время

\subsubsection*{Условие нормировки}

\[
\int\limits^{\infty} \Psi(\vec r, t) \Psi^*(\vec r, t) dV = 1
\]

\subsubsection*{Волновая функция основного состояния атома водорода}

\[
\Psi(\vec r, t) = \frac{1}{\sqrt{\pi a_b^3}} e^{-\frac{r}{a_b}}
e^{-\frac{i E_1 t}{\hbar}  }
\]

где $a_b$ - боровский радиус, $E_1$ - энергия первого уровня

\subsubsection*{Квантовый оператор}
Это функция, которая ставит в соответствие волновой функции какую-нибудь другую функцию

\[
\widehat{F} [\psi(\vec r, t)] = \Phi(\vec r, t)
\]

\subsubsection*{Среднее значение оператора}

\[
\overline{f} = \langle \psi | \Phi \rangle = \int \psi^*(\vec r, t) \Phi(\vec r, t) dV
\]

\subsubsection*{Постулаты квантовой механики}
\begin{itemize}
\item Значение волновой функции однозначно описывает состояние системы
\item Физическим величинам сопоставляеются линейные операторы, которые действуют на волновую функцию и в общем случае изменяют ее
\item {\bf (Принцип суперпозиции)} Если $\psi_1$ и $\psi_2$ - описывают физически реализуемые состояния квантовой системы, то $C_1\psi_1 + C_2\psi_2$ - тоже описывает физически реализуемое состояние такой системы.
\end{itemize}

\end{document} % Конец текста.
