\newpage
\section{Основные обозначения}
\begin{itemize}
    \item Множество натуральных чисел не включает в себя ноль: $ \N = \braces{1,\:2,\:\ldots} $.
    \item $\Z^+ \equiv \Z_+ := \braces{0,\:1,\:3,\:\ldots}$ --- множество положительных чисел.
    \item Назовём \textit{мультииндексом} вектор $\alpha = \brackets{\alpha_1,\alpha_2, \ldots\alpha_n},\: \alpha_i \in \Z^+ $.
          Для мультииндексов, как и для прочих векторов, вводится сумма, а также абсолютное значение (это не евклидова норма!):
          $\abs{\alpha} = \sum_{i = 1}^n \alpha_i$.
    \item Компактная запись оператора дифференцирования с использованием мультииндекса:
          \[
              \partial^{\alpha}_u = \frac{\partial^{\abs{\alpha}}u}
              {\prt^{\alpha_1}x_1 \prt^{\alpha_2}x_2 \ldots \prt^{\alpha_n}x_n},\:\:
              u = u\brackets{x_1, x_2, \dots x_n}.
          \]
    \item Если не оговорено, символы $\Omega, G \subset \R^n$ обычно обозначают области, $x \in \Omega$ --- переменные,
          $u = u(x)$ --- функцию нескольких переменных (решение уравнения).
    \item $C^0(\Omega),\: C^1(\Omega),\: C^k(\Omega)$\: --- множества функций, непрерывных, гладких, $k$ раз гладких на
          области $\Omega$ соответственно. $A(\Omega)$ --- множество аналитичных на $\Omega$ функций, то есть представимых
          степенным рядом:
          \[
              A(\Omega) := \braces{f : \forall x \in \Omega \:\ \: f(x) = \sum_{k = 0}^{\infty} f_k \brackets{x - x_0}^k,\: x_0 \in \Omega}
          \]
    \item $L$ --- линейный дифференциальный оператор. Например,
          \[
              Lu = \sum_{i,j = 1}^n A_{ij}(x) \mixd{u}{x_i}{x_j} + \sum_{k = 1}^n B_k(x)\pard{u}{x_k} + C(x)u
          \]
          --- общий вид линейного дифференциального оператора второго порядка.
\end{itemize}

\newpage
\section{Линейные уравнения второго порядка}
\subsection{Классификация уравнения в двумерном случае}
    Рассмотрим уравнения следующего вида:
    \begin{align}
        & a(x, y)\pardd{u}{x} + 2b(x,y)\mixd{u}{x}{y} + c(x, y)\pardd{u}{y} = F\brackets{x, y, u, \nabla u}, \label{lin2} \\
        & a,\ b,\ c \in  C^2(\Omega), \:\: \Omega \subset \R^2, \notag \\
        & u(x,y) \in C^2(\Omega). \notag
    \end{align}
    Попытаемся найти замену $\xi = \xi(x, y),\ \eta = \eta(x, y)$, которая привела бы уравнения (по крайней мере, старшие производные) к более простому виду.
    Замена должна быть обратимой, поэтому по теореме о системе обратных функций потребуем в области $\Omega$ равенства
    \begin{equation*}
        J \equiv \begin{vmatrix}
            \xi_x & \xi_y \\
            \eta_x & \eta_y
        \end{vmatrix}
        \not= 0.
    \end{equation*}
    Кроме того, потребуем $\xi,\eta \in C^2(\Omega)$. Посмотрим, как преобразуются коэффициенты при старших производных при
    такой замене. Первые производные:
    \begin{align*}
        & u_x = u_{\xi}\xi_x + \uet \eta_x,\\
        & u_y = u_{\xi}\xi_y + \uet \eta_y.
    \end{align*}
    Вторые производные:
    \begin{equation*}
        u_{xx} = \prt_x u_x = \prt_x\brackets{u_{\xi}\xi_x + \uet \eta_x} = \uxi \xi_{xx} + \xi_x\pard{\uxi}{x} + 
        \uet \eta_{xx} + \eta_x \pard{\uet}{x}.
    \end{equation*}
    Опустим слагаемые $ \uxi \xi_{xx} $ и $ \uet \eta_{xx} $, так как мы рассматриваем преобразования коэффициентов только при старших
    производных:
    \begin{align*}
        & \xi_x\brackets{\xi_x \pard{\uxi}{\xi} + \eta_x \pard{\uxi}{\eta}} + \eta_x\brackets{\eta_x\pard{\uet}{\eta} + \xi_x \pard{\uet}{\xi}} + \cdots = \\
        & = \xi_x^2 u_{\xi\xi} - \eta_x^2u_{\eta\eta} + 2\xi_x\eta_x u_{\xi\eta} + \cdots
    \end{align*}
    Аналогично нетрудно получить остальные выражения:
    \begin{align*}
        & u_{yy} = \xi_y^2 \uxx + \eta_y^2 \uee + 2 \xi_y\eta_y \uxe + \cdots,\\
        & u_{xy} = \xi_x\xi_y \uxx + \brackets{\xi_x\eta_y + \xi_y\eta_x}\uxe + \eta_x\eta_y\uee + \cdots
    \end{align*}
    Таким образом, после подстановки $\xi$ и $\eta$ в (\ref{lin2}) получаем следующее уравнение:
    \begin{equation*}
        \tilde{a}(\xi, \eta) \uxx + 2\tilde{b}(\xi, \eta)\uxe + \tilde{c}(\xi, \eta)\uee = \tilde{F}\brackets{\xi, \eta, u, \nabla u}.
    \end{equation*}
    Преобразованные коэффициенты равны:
    \begin{align*}
        & \tilde{a} = a\cdot\xi_x^2 + 2b\cdot\xi_x\xi_y + c\cdot\xi_y^2,\\
        & \tilde{b} = a\cdot\xi_x\eta_x + b\brackets{\xi_x\eta_y +\xi_y\eta_x} + c\cdot\xi_y\eta_y,\\
        & \tilde{c} = a\cdot\eta_x^2 + 2b\cdot\eta_x\eta_y + c\cdot\eta_y^2.
    \end{align*}
    В случае, если коэффициенты $a, b, c$ уравнения (\ref{lin2}) постоянны, мы можем выбрать функции $\xi$ и $\eta$ так, чтобы приравнять коэффициенты
    $\tilde{a}$ и $\tilde{c}$ к нулю. То же можно сделать, рассматривая уравнение в конкретной точке $(x_0, y_0) \in \Omega$ и положив
    $a = a(x_0,y_0),\: b = \ldots$