\newpage
\section{Основные обозначения}
\begin{itemize}
    \item Множество натуральных чисел не включает в себя ноль: $ \N = \braces{1,\:2,\:\ldots} $.
    \item $\Z^+ \equiv \Z_+ := \braces{0,\:1,\:3,\:\ldots}$ --- множество положительных чисел.
    \item Назовём \textit{мультииндексом} вектор $\alpha = \brackets{\alpha_1,\alpha_2, \ldots\alpha_n},\: \alpha_i \in \Z^+ $.
          Для мультииндексов, как и для прочих векторов, вводится сумма, а также абсолютное значение (это не евклидова норма!):
          $\abs{\alpha} = \sum_{i = 1}^n \alpha_i$.
    \item Компактная запись оператора дифференцирования с использованием мультииндекса:
          \[
              \partial^{\alpha}_u = \frac{\partial^{\abs{\alpha}}u}
              {\prt^{\alpha_1}x_1 \prt^{\alpha_2}x_2 \ldots \prt^{\alpha_n}x_n},\:\:
              u = u\brackets{x_1, x_2, \dots x_n}.
          \]
    \item Если не оговорено, символы $\Omega, G \subset \R^n$ обычно обозначают области, $x \in \Omega$ --- переменные,
          $u = u(x)$ --- функцию нескольких переменных (решение уравнения).
    \item $C^0(\Omega),\: C^1(\Omega),\: C^k(\Omega)$\: --- множества функций, непрерывных, гладких, $k$ раз гладких на
          области $\Omega$ соответственно. $A(\Omega)$ --- множество аналитичных на $\Omega$ функций, то есть представимых
          степенным рядом:
          \[
              A(\Omega) := \braces{f : \forall x \in \Omega \:\ \: f(x) = \sum_{k = 0}^{\infty} f_k \brackets{x - x_0}^k,\: x_0 \in \Omega}
          \]
    \item $L$ --- линейный дифференциальный оператор. Например,
          \[
              Lu = \sum_{i,j = 1}^n A_{ij}(x) \mixd{u}{x_i}{x_j} + \sum_{k = 1}^n B_k(x)\pard{u}{x_k} + C(x)u
          \]
          --- общий вид линейного дифференциального оператора второго порядка.
\end{itemize}

\newpage
\section{Квазилинейные уравнения второго порядка}
\subsection{Классификация уравнений двух независимых переменных}
    Рассмотрим уравнения следующего вида:
    \begin{align}
        & a(x, y)\pardd{u}{x} + 2b(x,y)\mixd{u}{x}{y} + c(x, y)\pardd{u}{y} = F\brackets{x, y, u, \nabla u}, \label{lin2} \\
        & a,\ b,\ c \in  C^2(\Omega), \:\: \Omega \subset \R^2, \notag \\
        & u(x,y) \in C^2(\Omega). \notag
    \end{align}
    В этой записи символ $\nabla u$ обозначает производные младших порядков.
    Такие уравнения называют квазилинейными, имея в виду, что они линейны только относительно производных старших порядков.
    Выберем точку $(x_0, y_0) \in \Omega$ и попытаемся найти замену $\xi = \xi(x, y),\ \eta = \eta(x, y)$, которая
    привела бы уравнения (по крайней мере, старшие производные) к более простому виду в некоторой окрестности $U$ этой точки.
    Замена должна быть обратимой, поэтому по теореме о системе обратных функций потребуем в $U$ равенства
    \begin{equation*}
        J \equiv \begin{vmatrix}
            \xi_x & \xi_y \\
            \eta_x & \eta_y
        \end{vmatrix}
        \not= 0.
    \end{equation*}
    Кроме того, потребуем $\xi,\eta \in C^2(U)$. Посмотрим, как преобразуются коэффициенты при старших производных при
    такой замене. Первые производные:
    \begin{align*}
        & u_x = u_{\xi}\xi_x + \uet \eta_x,\\
        & u_y = u_{\xi}\xi_y + \uet \eta_y.
    \end{align*}
    Вторые производные:
    \begin{equation*}
        u_{xx} = \prt_x u_x = \prt_x\brackets{u_{\xi}\xi_x + \uet \eta_x} = \uxi \xi_{xx} + \xi_x\pard{\uxi}{x} + 
        \uet \eta_{xx} + \eta_x \pard{\uet}{x}.
    \end{equation*}
    Опустим слагаемые $ \uxi \xi_{xx} $ и $ \uet \eta_{xx} $, так как мы рассматриваем преобразования коэффициентов только при старших
    производных:
    \begin{align*}
        & \xi_x\brackets{\xi_x \pard{\uxi}{\xi} + \eta_x \pard{\uxi}{\eta}} + \eta_x\brackets{\eta_x\pard{\uet}{\eta} + \xi_x \pard{\uet}{\xi}} + \cdots = \\
        & = \xi_x^2 u_{\xi\xi} - \eta_x^2u_{\eta\eta} + 2\xi_x\eta_x u_{\xi\eta} + \cdots
    \end{align*}
    Аналогично нетрудно получить остальные выражения:
    \begin{align*}
        & u_{yy} = \xi_y^2 \uxx + \eta_y^2 \uee + 2 \xi_y\eta_y \uxe + \cdots,\\
        & u_{xy} = \xi_x\xi_y \uxx + \brackets{\xi_x\eta_y + \xi_y\eta_x}\uxe + \eta_x\eta_y\uee + \cdots
    \end{align*}
    Таким образом, после подстановки $\xi$ и $\eta$ в (\ref{lin2}) получаем следующее уравнение:
    \begin{equation*}
        \tilde{a}(\xi, \eta) \uxx + 2\tilde{b}(\xi, \eta)\uxe + \tilde{c}(\xi, \eta)\uee = \tilde{F}\brackets{\xi, \eta, u, \nabla u}.
    \end{equation*}
    Преобразованные коэффициенты равны:
    \begin{align}
        & \tilde{a} = a\cdot\xi_x^2 + 2b\cdot\xi_x\xi_y + c\cdot\xi_y^2,\label{char_xi}\\
        & \tilde{b} = a\cdot\xi_x\eta_x + b\brackets{\xi_x\eta_y +\xi_y\eta_x} + c\cdot\xi_y\eta_y, \notag \\
        & \tilde{c} = a\cdot\eta_x^2 + 2b\cdot\eta_x\eta_y + c\cdot\eta_y^2. \label{char_eta}
    \end{align}
    Здесь $a = a(x, y)$, $b = b(x, y)$ и $c = c(x, y)$. Исключим возможный случай вырождения (\ref{lin2}) в уравнение первого порядка: выберем точку
    $(x_0, y_0) \in \Omega$ таким образом, чтобы $a$, $b$ и $c$ не обратились в ней
    (а значит и в $U$) в нуль одновременно. Чтобы упростить уравнение, мы потребуем, чтобы в этой окрестности замена $\xi(x, y)$, $\eta(x,y)$ обратила
    в нуль коэффициенты $\tilde{a}$ и $\tilde{c}$. Без ограничения общности мы можем считать, что $a \not= 0$ в окрестности точки $(x_0, y_0)$:
    если $a = 0$, то либо $c \not= 0$, и поменяв местами $x$ и $y$ мы получим, что коэффициент при $u_{xx}$ не равен нулю; либо $c = 0$, но тогда
    $b \not= 0$ --- уже единственный ненулевой коэффициент, чего мы и потребовали от замены.
    
    Обозначим $\lambda_1(x,y) := \frac{\xi_x}{\xi_y},\: \lambda_2(x, y) := \frac{\eta_x}{\eta_y}$. Подставим $\lambda_{1,2}$ в (\ref{char_xi})
    и (\ref{char_eta}) и приравняем полученные выражения к нулю:
    \begin{equation*}
        a \lambda^2 + 2b\lambda + c = 0.
    \end{equation*}
    Это соотношение называется \textit{характеристическим уравнением} (\ref{lin2}). Тип уравнения в рассматриваемой окрестности определяется дискриминантом
    его характеристического уравнения $D = b^2 - ac$:
    \begin{enumerate}
        \item Гиперболический тип при $D > 0$;
        \item Параболический тип при $D = 0$;
        \item Эллиптический тип при $D < 0$.
    \end{enumerate}

    Рассмотрим отдельно каждый случай.

    \textit{Гиперболический тип}. В этом случае характеристическое уравнение имеет два решения
    \begin{equation*}
        \lambda_{1,2} = \frac{-b \pm \sqrt{D}}{a},
    \end{equation*}
    откуда следуют уравнения на функции $\xi$ и $\eta$:
    \begin{align}
        & \xi_x - \lambda_1(x, y) \xi_y = 0,\notag \\ \label{char_eq_lin2}
        & \eta_x - \lambda_2(x, y) \eta_y = 0.
    \end{align}
    \begin{Def}
        Пусть задана некоторая функция $\phee(x, y),\: \phee \in C^1$. Множество $L = \braces{(x, y) \bigm| \phee(x, y) = 0}$ называется 
        \textit{характеристической линией}, или \textit{характеристикой} уравнения (\ref{lin2}), если $\forall (x, y)\in~L$ выполнено:
        \begin{itemize}
            \item $\grad \phee(x, y) \not= 0$;
            \item $a(x,y)\phee_x^2 + 2b(x,y)\phee_x\phee_y + c(x,y)\phee_y^2 = 0$.
        \end{itemize}
    \end{Def}

    Если мы предположим, что в окрестности $U$ существуют решения уравнений (\ref{char_eq_lin2}), удовлетворяющие условиям
    $\grad \xi \not= 0$, $\grad \eta \not= 0$ в $U$, то, согласно данному определению, кривые $\xi(x, y) = C_1$ и $\eta(x, y) = C_2$
    задают два семейства характеристик уравнения (\ref{lin2}). Приведём без доказательства теорему, позволяющую перейти от уравнений
    (\ref{char_eq_lin2}) в частных производных к обыкновенным дифференциальным уравнениям.
    \begin{theorem}
        Пусть задана функция $\phee(x, y),\: \phee \in C^2$. Семейство кривых $\phee(x,y)=C$ задаёт характеристики уравнения (\ref{lin2}) тогда и
        только тогда, когда $\phee(x, y) = C$ является общим решением (интегралом) одного из уравнений
        \begin{align*}
            & \frac{dy}{dx} = \lambda_1(x, y),\\
            & \frac{dy}{dx} = \lambda_2(x, y).
        \end{align*}
        Эти уравнения называют \textit{уравнениями характеристик}.
    \end{theorem}
    Например, для того, чтобы найти замену $\xi = \xi(x,y)$, необходимо решить следующую задачу Коши для уравнения характеристик:
    \begin{align*}
        & \frac{dy}{dx} = \lambda_1(x, y) \\
        & y \bigm|_{x = x_0} = C_1
    \end{align*}
    Найденное решение $y = y(x, C_1)$ по теореме о неявной функции эквивалентно уравнению $C_1 = \phee(x, y)$, задающему первое семейство
    характеристик уравнения (\ref{lin2}), а $\xi = C_1(x, y)$ будет искомой заменой. Аналогично находится $\eta = \psi(x, y)$.
    После подстановки уравнение сохранит второй порядок в силу обратимости замены, откуда следует, что $\tilde{b} \not= 0$ в некоторой
    окрестности $\tilde{U}$ --- образе окрестности $U$ при замене. Поэтому мы можем поделить уравнение на $2\tilde{b}$ и привести его к
    следующему виду, называемому \textit{каноническим}:
    \begin{equation}
        u_{\xi\eta} = \Phi(\xi, \eta, u, \nabla_{\xi\eta}u). \label{hyp_canon}
    \end{equation}
    Найдём общее решение (\ref{hyp_canon}) для важного частного случая $\uxe = 0$:
    \begin{align*}
        & \uxi(\xi, \eta) = \int \uxe d\eta = h(\xi), \:\: h \in C^1\\
        & u(\xi, \eta) = \int \uxi d\xi = f(\xi) + g(\eta), \:\: f,g \in C^2.
    \end{align*}
    Возвращаясь к исходным переменным:
    \begin{equation}
        u(x, y) = f\brackets{\phee(x,y)} + g\brackets{\psi(x, y)}, \label{waveform_gen_sol}
    \end{equation}
    где $f,g$ --- произвольные дважды гладкие функции одной переменной.
