\newpage
\subsection{Уравнения произвольного числа переменных}
    Рассмотрим линейные уравнения второго порядка следующего вида:
    \begin{align}
        & Lu = \sum_{i,j = 1}^n A_{ij}(x)\mixd{u}{x_i}{x_j} = F(x, u, \nabla u), \label{lin2n} \\ \notag
        & x \in \Omega \subset \R^n,\\ \notag
        & A_{ij}, u \in C^2(\Omega).
    \end{align}
    Предположим, что найдётся такая линейная замена вида
    \begin{equation}
        \xi_k = \sum_{i = 1}^n a_{ki} x_k, \label{lin_transform}
    \end{equation}
    приводящая уравнение (\ref{lin2n}) к более простому виду; например, обнуляющая все коэффициенты при смешанных производных.
    Посмотрим, как преобразуются коэффициенты: обозначим $u_{x_k} = \pard{u}{x_k}$, тогда
    \begin{align*}
        & u_{x_k} = \sum_{l = 1}^n \pard{u}{\xi_l} \pard{\xi_l}{x_k} := \sum_{l = 1}^n a_{lk}u_{\xi_l},\\
        & u_{x_i x_j} = \partial_j u_{x_i} = \partial_j \brackets{\sum_{l = 1}^n u_{x_l}a_{li}} = \sum_{l,k = 1}^n a_{li} a_{kj} u_{\xi_k \xi_l},\\
        & Lu = \sum_{i, j = 1}^n \brackets{\sum_{i,j = 1}^n A_{ij}(x) a_{li} a_{kj}} u_{\xi_k \xi_l} := \sum_{k, l = 1}^n\tilde{A}_{lk} u_{\xi_k \xi_l}.
    \end{align*}
    Если обозначить матрицами $A := \Vert A_{ij} \Vert$, $a := \Vert a_{kj} \Vert$, то можно записать следующее равенство:
    \begin{equation*}
        \tilde{A}_{kl} = a A a^T.
    \end{equation*}
    Уравнение в преобразованных координатах примет вид
    \begin{equation*}
        L= \ \sum_{k, l}^n \tilde{A}_{kl} \mixd{u}{\xi_k}{\xi_l} = \tilde{F}\brackets{\xi, u, \nabla_{\xi} u}.
    \end{equation*}
    Оператор $Lu = \sum_{i,j = 1}^n A_{ij}u_{x_i x_j}$ по своей структуре аналогичен алгебраической квадратичной форме. В самом деле, если $u \in C^2$,
    то порядок дифференцирования не важен, как и порядок сомножителей в слагаемых квадратичной формы: $u_{x_i x_j} = u_{x_j x_i}$. Вследствие этого каждая
    пара коэффициентов $A_{ij}, A_{ji},\: i \not= j$, соответствующих недиагональным элементам матрицы $A$, определена с точностью до постоянной.
    Поэтому мы можем выбрать $A_{ji} = A_{ij}$, чтобы матрица $A$ оказалась симметричной и задавала бы некоторую квадратичную форму
    $\sum_{i, j = 1}^n A_{ij} y_i y_j = \brackets{Ay,\ y}$. Вспомним теоремы из линейной алгебры о приведении квадратичной формы к каноническому виду.

    \begin{theorem}
        Пусть задана квадратичная форма $k(y) = \brackets{Ay,\ y}$. При замене координат $y_k = \sum_{i=1}^n b_{ki}\eta_i$, или, в матричном виде,
        $y = B\eta$, квадратичная форма принимает вид $\tilde{k}(\eta) = \brackets{AB\eta,\ \eta} = \brackets{B^T AB\eta,\ \eta}$. Матрица $A$ переходит
        в матрицу $\tilde{A} = B^T AB$.
    \end{theorem}

    \begin{theorem}
        Для всякой квадратичной формы $k(y) = \brackets{Ay,\ y}$ существует единственное \textit{ортогональное} преобразование $B: \: y = B\eta$, которое
        приводит её к виду $\tilde{k}(\eta) = \lambda_1 \eta_1^2 + \cdots + \lambda_n \eta_n^2$, где $\lambda_1,\ldots\lambda_n$ --- собственные числа
        матрицы $A$ с учётом кратности, то есть корни уравнения $\det \brackets{A - \lambda E} = 0$. Матрица $A$ переходит в матрицу
        $\tilde{A} = B^T AB = diag(\lambda_1,\ldots\lambda_n)$.
    \end{theorem}

    %\begin{theorem}
    %    Для всякой квадратичной формы $k(y) = \sum_{i, j = 1}^n A_{ij}y_i y_j$ существует невырожденное преобразование $B: \: y = B\eta$, которое
    %    приводит её к каноническому виду:
    %    \begin{equation*}
    %        \tilde{k}(\eta) = \sum_{i = 1}^p \eta_i^2 - \sum_{i = p + 1}^{p + m} \eta_i^2.
    %    \end{equation*}
    %\end{theorem}

    \begin{theorem}
        (Закон инерции квадратичных форм). Пусть в каноническом виде квадратичная форма записывается как
        \begin{equation*}
            \tilde{k}(\eta) = \lambda_1 \eta_1^2 + \cdots + \lambda_n \eta_n^2 = \sum_{i = 1}^n\lambda_i \eta_i^2 = 
            \sum_{j:\: \lambda_j > 0}^p |\lambda_j| \eta_j - \sum_{k:\: \lambda_k < 0}^m |\lambda_k| \eta_k.
        \end{equation*}
        Числа $p$ и $m$ --- положительный и отрицательный индексы квадратичной формы --- не зависят от преобразования $B$, которое осуществляет
        переход к каноническому виду.
    \end{theorem}
    \begin{note}
        Равенство $p + m = n$ не обязано выполняться. Число $p + m$ ненулевых коэффициентов в каноническом виде равно рангу $r$ квадратичной формы.
    \end{note}
    
    Из приведённых теорем следует, что существует такое ортогональное преобразование с матрицей $B$, что при замене $x = B\xi$ оператор
    $Lu = \sum_{i,j = 1}^n A_{ij}u_{x_i x_j}$ с матрицей $A$ преобразуется в оператор $\tilde{L}u = \sum_{k = 1}^n \tilde{A}_{kk} u_{\xi_k \xi_k}$
    с диагональной матрицей $\tilde{A} = B^T AB$. Чтобы выразить новые переменные через старые, как в (\ref{lin_transform}), воспользуемся ортогональностью
    оператора $B$:
    \begin{equation*}
        B^{-1} = B^T \: \Rightarrow \: \xi = B^T x.
    \end{equation*}
    Далее, при помощи преобразования масштабирования матрицу $\tilde{A}$ всегда можно нормировать, т.е. привести к диагональной матрице $\hat{A}$, диагональные элемнты которой $\hat{A}_i \in \braces{-1,0,1}$. Оператор $\tilde{L}$ при этом переходит в оператор $\Lambda$:
    \begin{equation}
        \Lambda u = \sum_{i = 1}^p \pardd{u}{\zeta} - \sum_{i = p+1}^{p+m} \pardd{u}{\zeta}. \label{2ord_canon}
    \end{equation}
\newpage
    Таким образом, была доказана
    \begin{theorem}
        Для любого линейного уравнения вида (\ref{lin2n}) в точке $x_0 \in \Omega$ существует замена координат $x \rightarrow \zeta$,
        приводящая уравнение к виду (\ref{2ord_canon}), называемому каноническим, причём $\zeta = MB^T x$, где $M$ --- диагональная матрица
        (нормировочная), а $B$ --- ортогональная матрица. Кроме того, числа $p$ и $m$ в записи (\ref{2ord_canon}) не зависят от замены.
    \end{theorem}
    Осталось привести классификацию линейных уравнений второго порядка в точке.
    \begin{Def}
        Говорят, что уравнение (\ref{lin2n}) в точке $x_0 \in \Omega$ имеет
        \begin{itemize}
            \item Эллиптический тип, если в канонической записи $p = n,\: m = 0$ или $m = n,\: p = 0$ (все слагаемые одного знака);
            \item Гиперболический тип, если $p = n-1,\: m = 1$ или $p = 1,\: m = n-1$ (все слагаемые за исключением единственного --- одного знака);
            \item Параболический тип, если $p + m \not= n$ (есть хотя бы одно нулевое слагаемое);
            \item Ультрагиперболический тип, если $p + m = n$, но и $p > 1$, и $m > 1$ --- есть несколько слагаемых разных знаков.
        \end{itemize}
    \end{Def}