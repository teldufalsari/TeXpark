\newpage
\section{Уравнение колебаний струны}
\subsection{Однородная задача}
    \begin{Def}
        Оператором Д'Аламбера называют следующий формальный оператор:
        \begin{equation*}
            \Box_a := \pardd{}{t} - a^2\pardd{}{x}.
        \end{equation*}
        При $a = 1$ индекс опускают:
        \begin{equation*}
            \Box_1 = \Box = \pardd{}{t} - \pardd{}{x}.
        \end{equation*}
    \end{Def}

    \begin{Def}
        Уравнением колебания струны называют следующее уравнение второго порядка:
        \begin{align*}
            u_{tt} - a^2 u_{xx} = f(t, x), \\
            f \in C^1,\: a \in \R,\: u \in C^2.
        \end{align*}
    \end{Def}
    Вначале ограничимся рассмотрением задачи Коши для однородного уравнения и для неограниченной ($x \in \R$) струны:
    \begin{align*}
        & u_{tt} - a^2 u_{xx} = 0,\: x \in \R;\\
        & u_{t=0} = \phee(x),\\
        & u'_t|_{t=0} = \psi(x).
    \end{align*}
    Характеристическое уравнение (\ref{char_2ord}) в нашем случае выглядит как
    \begin{equation*}
        dx^2 - a^2\cdot dt^2 = 0 \: \leftrightarrow \: \brackets{dx - a\cdot dt}\brackets{dx + a\cdot dt} = 0.
    \end{equation*}
    Решения этого уравнения:
    \begin{equation*}
        x = \pm at + C;
    \end{equation*}
    Семейства характеристик:
    \begin{equation*}
        x + at = C_1,\:\: x - at = C_2.
    \end{equation*}
    Наконец, искомая замена переменных:
    \begin{equation*}
        \begin{cases}
            \xi = x + at,\\
            \eta = x - at.
        \end{cases}
    \end{equation*}
    Преобразованные производные равны
    \begin{align*}
        & u_x = \uxi \xi_x + \uet \eta_x = \uxi + \uet,\\
        & u_t = a\uxi - a\uet,\\
        & u_{xx} = \prt_x\brackets{\uxi + \uet} = \uxx \xi_x + \uxe \eta_x + \uxe \xi_x + \uee \eta_x = \uxx + 2\uxe + \uee,\\
        & u_{tt} = a\prt_t \brackets{\uxi - \uet} = a^2\brackets{\uxx - 2\uxe + \uee}.
    \end{align*}
    Преобразованное уравнение:
    \begin{equation*}
        4a^2\uxe = 0 \: \Leftrightarrow \: \uxe = 0.
    \end{equation*}
    Как было показано ранее, общее решение этого уравнения имеет вид (\ref{waveform_gen_sol}):
    \begin{equation*}
        u(t,x) = f(\xi) + g(\eta) = f(x + at) + g(x - at).
    \end{equation*}

    Теперь найдём решение задачи Коши:
    \begin{numcases}{}
        f (x) + g (x) = \phee(x),\notag \\ 
        af' (x) - ag' (x) = \psi (x).\label{kluvik}   
    \end{numcases}
    Интегрируя (\ref{kluvik}) по $x$:
    \begin{equation*}
        f(x) - g(x) = \frac 1a \int\limits_0^x \psi(s)ds + C.
    \end{equation*}
    Тогда, вместе с первым уравнением системы, находим решение:
    \begin{align*}
        & f(x) = \frac 12 \phee(x) + \frac{1}{2a} \int\limits_0^x \psi(s)ds + C,\\
        & g(x) = \frac 12 \phee(x) - \frac{1}{2a} \int\limits_0^x \psi(s)ds - C.
    \end{align*}
    Тогда
    \begin{align}
        & u(t, x) = \frac{1}{2}\brackets{\phee(x + at) + \phee(x - at)} + 
        \frac{1}{2a}\brackets{\int\limits_0^{x + at} \psi(s)ds - \int\limits_0^{x - at} \psi(s)ds} = \notag \\
        & = \frac{1}{2}\brackets{\phee(x + at) + \phee(x - at)} + \frac{1}{2a} \int\limits_{x - at}^{x + at} \psi(s)ds \label{DAlembert_homogenous}
    \end{align}
    Соотношение (\ref{DAlembert_homogenous}) называется \textit{формулой Д'Аламбера} для однородного уравнения.

\if 0 %##################################################################################################
    Докажем, что найденное решение задачи Коши единственное: предположим, что нашлись две функции $u_1, u_2 \in C^2$, удовлетворяющие
    условиям
    \begin{equation*}
        \begin{cases}
            \Box_a u_1 = 0\\
            u_1|_{t=0} = \phee(x)\\
            u_1'|_{t=0} = \psi(x)
        \end{cases}
        \:\:
        \begin{cases}
            \Box_a u_2 = 0\\
            u_2|_{t=0} = \phee(x)\\
            u_2'|_{t=0} = \psi(x).
        \end{cases}
    \end{equation*}
    Возьмём $u = u_1 - u_2,\: u \in C^2$. Тогда
    \begin{align*}
        & \Box_a u = \Box_a u_1 - \Box_a u_2 = 0,\\
        & u|_{t = 0} = \phee - \phee = 0,\\
        & u'|_{t=0} = \psi - \psi = 0.
    \end{align*}
\fi %####################################################################################################
\newpage
\subsection{Неоднородное уравнение. Принцип Дюамеля}
    \begin{theorem}
        (принцип Дюамеля) Задача Коши для неоднородного волнового уравнения
        \begin{equation*}
            \begin{cases}
                u_{tt} - a^2 u_{xx} = f(t,x) \\
                u|_{t=0} = 0,\: u'_{t=0} = 0
            \end{cases}
        \end{equation*}
        имеет решение вида
        \begin{equation}
            u(t,x) = \int\limits_0^t v(\tau, t, x) d\tau, \label{DuhamelSolution}
        \end{equation}
        где $v(t,x)$ --- решение следующей задачи Коши для однородного уравнения:
        \begin{equation}
            \begin{cases}
                v_{tt} - a^2 v_{xx} = 0\\
                v|_{t = \tau} = 0,\: v'_t|_{t = \tau} = f(\tau, x). \label{CauchyDuhamel}
            \end{cases}
        \end{equation}
    \end{theorem}
    \begin{proof}
        Найдём производные решения (\ref{DuhamelSolution}):
        \begin{align*}
            & u_t = \pard{}{t} \int\limits_0^t v(\tau, t, x) d\tau = \underbrace{v(t, t, x)}_{= v|_{\tau = t} = 0} + \int\limits_0^t v_t(\tau, t, x) d\tau,\\
            & u_{tt} = v_t(t, t, x) + \int\limits_0^t v_{tt}(\tau, t, x) d\tau = f(t, x) + \int\limits_0^t v_{tt}(\tau, t, x) d\tau,\\
            & u_{xx} = \int\limits_0^t v_{xx}(\tau, t ,x) d\tau.
        \end{align*}
        Проверим, удовлетворяет ли это решение уравнению колебаний:
        \begin{gather*}
            u_{tt} - a^2 u_{xx} = f(t, x) + \int\limits_0^t v_{tt}(\tau, t, x) d\tau - a^2 \int\limits_0^t v_{xx}(\tau, t ,x) d\tau = \\
            = f(t, x) + \int\limits_0^t \brackets{\Box_a u(\tau, t, x)} d\tau = f(t, x).
        \end{gather*}
        Условия Коши в таком случае выполнены очевидным образом.
        \textbf{А единственность Коши доказывать будет?}
    \end{proof}
    Обозначим $\hat{t} := t - \tau$, и сделаем замену переменных в (\ref{CauchyDuhamel}):
    \begin{equation*}
        \begin{cases}
            v_{\hat{t}\hat{t}} - a^2 v_{xx} = 0\\
            v|_{\hat{t} = 0} = 0,\: v'_{\hat{t}}|_{\hat{t} = 0} = f(\tau, x).
        \end{cases}
    \end{equation*}
    Решение такой задачи находится по формуле Д'Аламбера (\ref{DAlembert_homogenous}):
    \begin{equation*}
        v(\hat{t}, x) = \frac{1}{2a} \int\limits_{x - a\hat{t}}^{x + a\hat{t}} f(\tau, s) ds.
    \end{equation*}
    Переходя к старым переменным и подставляя решение в формулу (\ref{DuhamelSolution}), получаем это же решение, но уже в явном виде:
    \begin{equation*}
        u(t,x) = \frac{1}{2a} \int\limits_0^t d\tau \int \limits_{x - a(t - \tau)}^{x + a(t - \tau)} f(\tau, s) ds.
    \end{equation*}
